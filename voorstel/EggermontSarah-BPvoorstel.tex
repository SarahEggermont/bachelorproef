%==============================================================================
% Sjabloon onderzoeksvoorstel bachproef
%==============================================================================
% Gebaseerd op document class `hogent-article'
% zie <https://github.com/HoGentTIN/latex-hogent-article>

\documentclass{hogent-article}

% Invoegen bibliografiebestand
\usepackage[backend=biber,style=apa]{biblatex}
\DeclareLanguageMapping{dutch}{dutch-apa}
\addbibresource{voorstel.bib}

% Informatie over de opleiding, het vak en soort opdracht
\studyprogramme{Professionele bachelor toegepaste informatica}
\course{Bachelorproef}
\assignmenttype{Onderzoeksvoorstel}

\academicyear{2023-2024}

% TODO: Werktitel - nog aan te passen
\title{Gamification om werknemers aan het sporten te krijgen}

\author{Sarah Eggermont}
\email{sarah.eggermont@student.hogent.be}


% TODO: Geef de co-promotor op - meerdere?
\supervisor[Co-promotor]{G. Vande Maele (we are, \href{mailto:guillaume@we-are.be}{guillaume@we-are.be})}
\supervisor[Co-promotor]{M. De Buck (we are, \href{mailto:manu@we-are.be}{manu@we-are.be})}

% Binnen welke specialisatierichting uit 3TI situeert dit onderzoek zich?
% Kies uit deze lijst:
%
% - Mobile \& Enterprise development
% - AI \& Data Engineering
% - Functional \& Business Analysis
% - System \& Network Administrator
% - Mainframe Expert
% - Als het onderzoek niet past binnen een van deze domeinen specifieer je deze
%   zelf
%
\specialisation{Mobile \& Enterprise development}
% TODO: keywords aanvullen
\keywords{Gamification, responsive website}

\begin{document}

\begin{abstract}
  Hier schrijf je de samenvatting van je voorstel, als een doorlopende tekst van één paragraaf. Let op: dit is geen inleiding, maar een samenvattende tekst van heel je voorstel met inleiding (voorstelling, kaderen thema), probleemstelling en centrale onderzoeksvraag, onderzoeksdoelstelling (wat zie je als het concrete resultaat van je bachelorproef?), voorgestelde methodologie, verwachte resultaten en meerwaarde van dit onderzoek (wat heeft de doelgroep aan het resultaat?).
\end{abstract}

\tableofcontents

% De hoofdtekst van het voorstel zit in een apart bestand, zodat het makkelijk
% kan opgenomen worden in de bijlagen van de bachelorproef zelf.
%---------- Inleiding ---------------------------------------------------------

\section{Introductie}%
\label{sec:introductie}

% TODO: Cijfers over stilzitten, hoeveel een persoon per dag zou moeten staan
Bij volwassenen wordt een sedentaire leefstijl geassocieerd met schadelijke gevolgen voor de volgende gezondheidskwesties: sterfte in het algemeen, sterfte door hart- en vaatziekten en kanker, het voorkomen van hart- en vaatziekten, diabetes type 2 en kanker \autocite{Bull2020}. Het is dus van groot belang dat voldoende beweging een prioriteit is.

Volgens \textcite{Bull2020} moeten volwassenen tussen de 18 en 64 jaar oud, wekelijks 150 à 300 minuten sporten met gemiddelde intensiteit, of 75 à 150 minuten met krachtige intensiteit. Voor mensen met een beperking worden dezelfde hoeveelheden sport aangeraden, hoewel daar mogelijks samen met een medisch verantwoordelijke bekeken moet worden in welke mate dit mogelijk is, afhankelijk van de beperking. Voor zwangere of net bevallen vrouwen wordt er minstens 150 minuten per week, met gemiddelde intensiteit, aangeraden.

Daarnaast gaan verminderde fysieke activiteit ook gepaard met meer depressie-, angst- en stresssymptomen \autocite{Stanton2020}.

Wereldwijd beschouwt \autocite{Hallal2012} 31,1\% van de bevolking als inactief. Dit wil zeggen dat, op het moment van onderzoek, bijna een derde van de volwassen wereldbevolking de vooropgestelde aanbevelingen van ``World Health Organization'' (WHO) niet haalt. Voor Europa ligt deze waarde zelfs op 34,8\%.

% TODO: Verbeteren productiviteit door sporten?
Om deze problematiek te proberen verhelpen, zal een sportplatform ontwikkeld worden. Deze paper onderzoekt hoe gamification hierbij kan helpen. Gamification is in de literatuur beschreven als het gebruiken van spelelementen in een niet-spelgerelateerde context \autocite{Gaalen2020}.


% TODO: beschrijving van literatuurstudie
In de literatuurstudie zal allereerst het concept van gamification uiteengezet worden. Ten tweede worden de populairste technieken, sociale aspecten en de invloed van gamification besproken. Daarna zal het effect dat beweging en een gezonde levensstijl heeft op productiviteit geïllustreerd worden. Ten slotte wordt er gekeken naar bestaande sportapplicaties en -platformen, en beschreven welke vormen van gamification daar in voorkomen.

Nadien zal in de methodologie worden uiteengezet welke stappen er nodig zijn om tot het sportplatform te komen dat, door middel van gamification, werknemers van \textbf{bedrijfX} en \textbf{bedrijfY} motiveert om meer te sporten.

Uiteindelijk kunnen de conclusies van het onderzoek teruggevonden worden.

% Waarover zal je bachelorproef gaan? Introduceer het thema en zorg dat volgende zaken zeker duidelijk aanwezig zijn:

% \begin{itemize}
%    \item kaderen thema
%    \item de doelgroep
%    \item de probleemstelling en (centrale) onderzoeksvraag
%    \item de onderzoeksdoelstelling
% \end{itemize}

% Denk er aan: een typische bachelorproef is \textit{toegepast onderzoek}, wat betekent dat je start vanuit een concrete probleemsituatie in bedrijfscontext, een \textbf{casus}. Het is belangrijk om je onderwerp goed af te bakenen: je gaat voor die \textit{ene specifieke probleemsituatie} op zoek naar een goede oplossing, op basis van de huidige kennis in het vakgebied.

% De doelgroep moet ook concreet en duidelijk zijn, dus geen algemene of vaag gedefinieerde groepen zoals \emph{bedrijven}, \emph{developers}, \emph{Vlamingen}, enz. Je richt je in elk geval op it-professionals, een bachelorproef is geen populariserende tekst. Eén specifiek bedrijf (die te maken hebben met een concrete probleemsituatie) is dus beter dan \emph{bedrijven} in het algemeen.

% Formuleer duidelijk de onderzoeksvraag! De begeleiders lezen nog steeds te veel voorstellen waarin we geen onderzoeksvraag terugvinden.

% Schrijf ook iets over de doelstelling. Wat zie je als het concrete eindresultaat van je onderzoek, naast de uitgeschreven scriptie? Is het een proof-of-concept, een rapport met aanbevelingen, \ldots Met welk eindresultaat kan je je bachelorproef als een succes beschouwen?

%---------- Stand van zaken ---------------------------------------------------

\section{Literatuurstudie}%
\label{sec:state-of-the-art}

% Hier beschrijf je de \emph{state-of-the-art} rondom je gekozen onderzoeksdomein, d.w.z.\ een inleidende, doorlopende tekst over het onderzoeksdomein van je bachelorproef. Je steunt daarbij heel sterk op de professionele \emph{vakliteratuur}, en niet zozeer op populariserende teksten voor een breed publiek. Wat is de huidige stand van zaken in dit domein, en wat zijn nog eventuele open vragen (die misschien de aanleiding waren tot je onderzoeksvraag!)?

\subsection{Gamification}

Volgens \textcite{Deterding2011} is gamification te beschrijven als het gebruiken van speldesignelementen in een niet-spelgerelateerde context. Gamification bestaat uit drie hoofdonderdelen: de gebruikte techniek, de psychologische uitkomsten en de verdere invloed op het gedrag \autocite{Hamari2014}. Daarnaast zijn sociale aspecten ook essentieel: door het ontstaan van een competitie streven mensen ernaar erkenning te ontvangen \autocite{Hamari2013}.

\subsubsection{Populairste technieken}
Volgens \textcite{Hamari2014} zijn punten, scoreborden en vooropgestelde uitdagingen de drie meest voorkomende technieken. Daarnaast komen ook het gebruik van levels \autocite{Dong2012}, beloningen \autocite{Flatla2011} en een overzicht van vooruitgang of het bekomen van badges \autocite{Li2012} veelvuldig voor.

\subsubsection{Sociale aspecten van gamification}
Sociale invloed verwijst naar de perceptie van een individu over het belang dat anderen hechten aan een bepaald doelgedrag en of ze verwachten dat iemand dat gedrag zal vertonen \autocite{Ajzen1991}.

Herkenning beschrijft de sociale feedback die gebruikers krijgen op hun gedrag \autocite{Cheung2011}. \textcite{Hamari2013} suggereren dat het ontvangen van erkenning, een bepaalde bereidheid creëert om anderen binnen eenzelfde dienst wederzijds te erkennen, wat de sociale interactie verder bevordert.

Beide aspecten zijn belangrijk om in acht te nemen wanneer gamification geïmplementeerd wordt. Het is volgens \textcite{Preece2001} namelijk zo dat een service positiever wordt ervaren wanneer het een gevoel van erkenning door andere gebruikers oplevert. Dit zal er op zijn beurt voor zorgen dat de houding van de gebruiker ten opzichte van de dienst positief beïnvloed wordt.

\subsubsection{Invloed van gamification}
Op dit moment is vooral de invloed die gamification heeft op het gedrag van de gebruiker onderzocht. Wanneer psychologische gevolgen ook bevraagd zijn, wordt er vooral gefocust op motivatie, attitude en plezier \autocite{Hamari2014}. Studies van \textcite{Hamari2013a} hebben aangetoond dat de resultaten van gamification mogelijks niet voor alle gebruikers op lange termijn doeltreffend zijn.

\subsection{Beweging en productiviteit}



\subsection{Gamification in bestaande sportapplicaties}


% Zijn er al gelijkaardige onderzoeken gevoerd? Wat concluderen ze? Wat is het verschil met jouw onderzoek?

% Verwijs bij elke introductie van een term of bewering over het domein naar de vakliteratuur, bijvoorbeeld~\autocite{Hykes2013}! Denk zeker goed na welke werken je refereert en waarom \autocite{Bitrian2020}.

% Draag zorg voor correcte literatuurverwijzingen! Een bronvermelding hoort thuis \emph{binnen} de zin waar je je op die bron baseert, dus niet er buiten! Maak meteen een verwijzing als je gebruik maakt van een bron. Doe dit dus \emph{niet} aan het einde van een lange paragraaf. Baseer nooit teveel aansluitende tekst op eenzelfde bron.

% Als je informatie over bronnen verzamelt in JabRef, zorg er dan voor dat alle nodige info aanwezig is om de bron terug te vinden (zoals uitvoerig besproken in de lessen Research Methods).

% Voor literatuurverwijzingen zijn er twee belangrijke commando's:
% \autocite{KEY} => (Auteur, jaartal) Gebruik dit als de naam van de auteur
%   geen onderdeel is van de zin.
% \textcite{KEY} => Auteur (jaartal)  Gebruik dit als de auteursnaam wel een
%   functie heeft in de zin (bv. ``Uit onderzoek door Doll & Hill (1954) bleek
%   ...'')

% Je mag deze sectie nog verder onderverdelen in subsecties als dit de structuur van de tekst kan verduidelijken.

%---------- Methodologie ------------------------------------------------------
\section{Methodologie}%
\label{sec:methodologie}

Het onderzoek zal beginnen met een literatuurstudie over gamification. Hierbij is het belangrijk om technieken te identificeren die van toepassing zijn op deze specifieke casus. De resultaten hiervan kunnen gebruikt worden in de volgende fase.

Deze volgende fase bestaat uit een bevraging bij IT-werknemers van \textbf{enkele} Gentse bedrijven. De bedoeling hiervan is enerzijds achterhalen in welke mate sport en beweging een rol speelt in hun dagelijkse leven op het moment van het interview, en anderzijds te weten komen in welke mate en op welke manieren deze werknemers gestimuleerd kunnen worden om meer te gaan sporten, aan de hand van een competitief sportplatform. Hier kan ook de theorie die verkregen is uit de literatuurstudie getoetst worden.

Steunend op de bekomen resultaten uit zowel de literatuurstudie als de analyse van de gebruikersinterviews, zal dan een applicatie ontwikkeld worden. Het platform zal bestaan in de vorm van een responsive \href{https://react.dev/}{React}-website, geschreven in \href{https://www.typescriptlang.org/}{TypeScript}, waarin sportgegevens van deelnemende werknemers worden verzameld aan de hand van applicaties en hun ''Application Programming Interface'' (API) zoals \href{https://developers.strava.com/}{Strava}, \href{https://dev.fitbit.com/}{Fitbit} en \href{https://developer.garmin.com/gc-developer-program/overview/}{Garmin Connect}. Deze gegevens zullen in combinatie met gamification gebruikt worden om een competitie tussen de deelnemende bedrijven op te zetten.

Deze POC zal dan dienen om de hypotheses te toetsen: heeft het competitief sportplatform, dat gebruik maakt van gamification, een positieve invloed op het sportgedrag van werknemers in een sedentaire job? Om deze vraag te beantwoorden zullen opnieuw gebruikersinterviews plaatsvinden, met dezelfde personen die voor aanvang van de ontwikkelingsfase geïnterviewd zijn en die voor een periode van enkele weken deze POC hebben kunnen gebruiken. Een grondige analyse van de sportgegevens op het platform en de antwoorden van deze ondervraging, leiden tot de conclusie van dit onderzoek.

% Hier beschrijf je hoe je van plan bent het onderzoek te voeren. Welke onderzoekstechniek ga je toepassen om elk van je onderzoeksvragen te beantwoorden? Gebruik je hiervoor literatuurstudie, interviews met belanghebbenden (bv.~voor requirements-analyse), experimenten, simulaties, vergelijkende studie, risico-analyse, PoC, \ldots?

% Uit dit onderdeel moet duidelijk naar voor komen dat je bachelorproef ook technisch voldoen\-de diepgang zal bevatten. Het zou niet kloppen als een bachelorproef informatica ook door bv.\ een student marketing zou kunnen uitgevoerd worden.

% Je beschrijft ook al welke tools (hardware, software, diensten, \ldots) je denkt hiervoor te gebruiken of te ontwikkelen.

% Probeer ook een tijdschatting te maken. Hoe lang zal je met elke fase van je onderzoek bezig zijn en wat zijn de concrete \emph{deliverables} in elke fase?

%---------- Verwachte resultaten ----------------------------------------------
\section{Verwacht resultaat}%
\label{sec:verwachte_resultaten}

De verworven kennis geeft momenteel aan dat een sportplatform met implementatie van gamification wel degelijk een positieve invloed zal hebben op het sportgedrag van gebruikers. Deze conclusie ... onder voorwaarde dat er voldoende aandacht besteed wordt aan het sociale aspect van gamification en de gebruikers ook een persoonlijke ontwikkeling kunnen opmerken op het platform.

% TODO: iets over productiviteit

% Hier beschrijf je welke resultaten je verwacht. Als je metingen en simulaties uitvoert, kan je hier al mock-ups maken van de grafieken samen met de verwachte conclusies. Benoem zeker al je assen en de onderdelen van de grafiek die je gaat gebruiken. Dit zorgt ervoor dat je concreet weet welk soort data je moet verzamelen en hoe je die moet meten.

% Wat heeft de doelgroep van je onderzoek aan het resultaat? Op welke manier zorgt jouw bachelorproef voor een meerwaarde?

% Hier beschrijf je wat je verwacht uit je onderzoek, met de motivatie waarom. Het is \textbf{niet} erg indien uit je onderzoek andere resultaten en conclusies vloeien dan dat je hier beschrijft: het is dan juist interessant om te onderzoeken waarom jouw hypothesen niet overeenkomen met de resultaten.



\printbibliography[heading=bibintoc]

\end{document}