%==============================================================================
% Sjabloon poster bachproef
%==============================================================================
% Gebaseerd op document class `a0poster' door Gerlinde Kettl en Matthias Weiser
% Aangepast voor gebruik aan HOGENT door Jens Buysse en Bert Van Vreckem

\documentclass[a0,portrait]{hogent-poster}

% Info over de opleiding
\course{Bachelorproef}
\studyprogramme{toegepaste informatica}
\academicyear{2023-2024}
\institution{Hogeschool Gent, Valentin Vaerwyckweg 1, 9000 Gent}

% Info over de bachelorproef
\title{Competitief sportplatform met gamification zodat werknemers in een sedentaire job meer bewegen}
\subtitle{Een proof of concept}
\author{Sarah Eggermont}
\email{sarah.eggermont@student.hogent.be}
\supervisor{Sebastiaan Labijn}
\cosupervisor{Guillaume Vande Maele (we are)}

% Indien ingevuld, wordt deze informatie toegevoegd aan het einde van de
% abstract. Zet in commentaar als je dit niet wilt.
\specialisation{Mobile \& Enterprise developer}
\keywords{Gamification, sportplatform, gezondheid}
% \projectrepo{https://github.com/user/repo}

\begin{document}

\maketitle

\begin{abstract}

Beweging speelt een grote rol in zowel de fysieke als de mentale gezondheid van mensen. Bijna één derde van de wereldbevolking beweegt te weinig en ondervindt hier vroeg of laat de nadelen van. Om die reden bespreekt dit onderzoek of een sportplatform, dat gebruik maakt van gamification, gebruikt kan worden om medewerkers van \href{https://en.joule.be/}{Joule}, \href{https://www.ventures4growth.com/en}{Ventures 4 Growth}, \href{https://www.mace-legal.com/}{mace}, \href{https://planetb.life/en}{PlanetB}, \href{https://www.we-are.be/}{we are} en \href{https://www.delaware.pro/en-be}{delaware}, die een sedentaire job beoefenen, aan te zetten om meer te sporten.

Na een literatuurstudie rond het belang van beweging, gamification en hoe gamification de intrinsieke motivatie om te bewegen kan bevorderen, worden werknemers van eerder genoemde bedrijven geïnterviewd om de succescriteria en de benodigdheden van het nieuwe platform te bepalen. Aan de hand van deze criteria wordt een platform, in de vorm van een responsive website, gecreëerd. In deze proof of concept wordt gamification geïmplementeerd en worden sportgegevens van deelnemende werknemers verzameld en grafisch voorgesteld op het platform. Tegelijkertijd wordt ook de beleving omtrent het gamification-aspect bevraagd. Deze gegevens leiden, samen met de eerder verzamelde data, tot de conclusie van dit onderzoek.

Dit onderzoek suggereert dat een sportplatform met gamification wel degelijk zorgt voor een verbetering in de hoeveelheid beweging van mensen in een sedentaire job. Daarnaast stelt het dat gamificationtechnieken zoals punten en scoreborden, vooral wanneer dit zorgt voor een onderlinge competitie, het meest succesvol zijn. Tenslotte zijn er in dit onderzoek geen gamificationtechnieken opgemerkt die een negatief effect hebben op de hoeveelheid beweging van gebruikers, maar moet wel opgemerkt worden dat, binnen de context van een sportapplicatie, personen het storend vinden om meldingen te ontvangen als deel van de gamification.

\end{abstract}

\begin{multicols}{2} % This is how many columns your poster will be broken into, a portrait poster is generally split into 2 columns

\section{Introductie}

Probleemstelling

Onderzoeksvragen

Bijdrage

\section{Methodologie}

Flowchart ervan

\section{Resultaten}

\begin{center}
  \captionsetup{type=figure}
  \includegraphics[width=1.0\linewidth]{grail}
  \captionof{figure}{He hasn't got shit all over him. The nose? Where'd you get the coconuts? What do you mean? We shall say `Ni' again to you, if you do not appease us}
\end{center}

\section{Conclusies}

Deze paper suggereert dat een competitief sportplatform, dat gebruik maakt van
gamification, een positieve invloed heeft op het sportgedrag van werknemers met een sedentaire job.

Op vlak van gamification, hebben punten en scoreborden het meeste succes, vooral wanneer dit zorgt voor een onderlinge competitie. Er zijn in dit onderzoek geen gamification technieken naar boven gekomen die een negatief effect hebben op de hoeveelheid beweging. Specifiek voor sportapplicaties, suggereert dit onderzoek echter wel dat gebruikers het storend vinden om meldingen te ontvangen als deel van de gamification. In welke mate dit een mogelijke beslissende factor is in het al dan niet gebruiken van een sportapplicatie, moet verder onderzoek uitwijzen.

\section{Beperkingen huidig onderzoek}

Gezien het beperkte aantal deelnemers, moeten de bekomen resultaten en conclusies met voorzichtigheid benaderd worden en kunnen ze niet veralgemeend worden voor alle personen met een sedentaire job.

\section{Toekomstig onderzoek}

Een eerste aanbeveling voor verder onderzoek, is het vergroten van de
steekproef. Daarnaast leiden ook de volgende onduidelijkheden of vragen tot verder onderzoek:

\begin{itemize}
    \item Zijn de resultaten verschillend wanneer er een onderscheid gemaakt kan worden tussen intensieve activiteiten en activiteiten met gemiddelde intensiteit?
    \item Geven personen die weinig maar wel intensief sporten enkel deze gegevens in, waardoor het lijkt dat ze minder doen dan mensen die regelmatig minder intensief bewegen? Of bewegen ze effectief weinig naast deze intensieve activiteiten?
    \item Welk effect heeft een sportplatform met gamification voor mensen die ouder zijn dan de deelnemers van dit onderzoek?
    \item Hoe willen mensen gemotiveerd worden als ze geen meldingen willen ontvangen?
\end{itemize}


\end{multicols}
\end{document}