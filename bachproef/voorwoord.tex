%%=============================================================================
%% Voorwoord
%%=============================================================================

\chapter*{\IfLanguageName{dutch}{Woord vooraf}{Preface}}%
\label{ch:voorwoord}

%% TODO:
%% Het voorwoord is het enige deel van de bachelorproef waar je vanuit je
%% eigen standpunt (``ik-vorm'') mag schrijven. Je kan hier bv. motiveren
%% waarom jij het onderwerp wil bespreken.
%% Vergeet ook niet te bedanken wie je geholpen/gesteund/... heeft

Met dit onderzoek wil ik voornamelijk onderzoeken hoe je mensen meer aan het bewegen krijgt. Ik ben zelf aan de start van mijn studies rugproblemen beginnen ervaren, en dan voornamelijk tijdens examenperiodes. Na even informeren, bleek dit volledig te wijten aan de vele uren dat ik zat tijdens examenperiodes. Na het bewust inplannen van beweegmomenten merkte ik hier echter verbetering in, en tot op de dag van vandaag probeer ik me hier zeer bewust van te blijven.

Nu ik stage gedaan heb, besef ik echter nog meer hoe moeilijk het kan zijn om voldoende te bewegen wanneer je een zittend beroep hebt. Die reden motiveerde me ook om specifiek te onderzoeken hoe deze doelgroep gemotiveerd en gesensibiliseerd kan worden.

Graag dank ik Dhr. Vande Maele voor zijn expertise binnen het domein, voor de hulp bij het vinden van testpersonen en voor het nalezen van deze bachelorproef. Daarnaast wil ik ook Dhr. Labijn bedanken voor het adviseren omtrent gepaste onderzoekstechnieken en het nalezen van deze scriptie. Door bijdrage van zowel Dhr. Vande Maele als Dhr. Labijn is mijn visie op de mogelijke richtingen waarnaar dit onderzoek kon gaan sterk verbreed. Hierbij hebben zij mij ook begeleid in het afbakenen van dit onderzoek, waarvoor dank.

Verder dank ik graag familie en vrienden voor alle steun tijdens het schrijven, alsook in het bijzonder mijn vriend Stijn Heyman, voor het meermaals doornemen en adviseren van mijn bachelorproef op vlak van vlotheid, samenhang en taal.

Tot slot dank ik u om interesse te tonen en de tijd te nemen dit onderzoek te lezen. Ik hoop dat ik u kan overtuigen om wat regelmatiger recht te staan, de fiets te nemen in plaats van de auto indien mogelijk en regelmatig te bewegen, hoe dit er voor u ook mag uitzien.