\chapter{\IfLanguageName{dutch}{Proof of concept}{Proof of concept}}%
\label{ch:proofofconcept}

\section{Gekozen framework}

Gezien de expertise van we are, is er voor het platform gekozen voor een responsive React\footnote{\href{https://react.dev/}{https://react.dev/}}-website.

Gezien de korte ontwikkeltijd die beschikbaar was, zijn Material UI Core\footnote{\href{https://mui.com/core/}{https://mui.com/core/}} en Material UI X\footnote{\href{https://mui.com/x/}{https://mui.com/x/}} component libraries gebruikt voor de ontwikkeling van deze website, om zo het ontwikkelproces te verkorten. Er is gekozen voor Material UI door diens universele look-and-feel, het ruime aanbod aan gratis componenten en de beschikbaarheid van grafische componenten.

\section{Opbouw van de applicatie}

De applicatie is als volgt opgesteld: een centraal dashboard, een overzicht van eigen prestaties en een oplijsting van de teamleden.

Op het dashboard is de meeste informatie te zien: zowel persoonlijke evolutie over één week en over drie weken als de team en persoonlijke ranking zijn er te vinden. Om kleinere teams evenveel kans te geven, houdt de teamranking rekening met de grote van het team bij de berekening.

Op het persoonlijk overzicht is een lijst van de gepresteerde activiteiten te zien, en kunnen gebruikers ook hun prestaties ingeven. Hier is voldoende ruimte voorzien voor de ingave van technische gegevens, zoals gemiddelde hartslag en hoogtemeters. Afhankelijk van het type activiteit, en wanneer een afstand en duur ingegeven zijn, berekent de applicatie ook een gemiddelde snelheid. Zo kunnen frequente sporters eventueel een vooruitgang opmerken.

\section{Mogelijke verbeteringen}