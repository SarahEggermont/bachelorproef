%===============================================================================
% LaTeX sjabloon voor de bachelorproef toegepaste informatica aan HOGENT
% Meer info op https://github.com/HoGentTIN/latex-hogent-report
%===============================================================================

\documentclass[dutch,dit,thesis]{hogentreport}

\usepackage{lipsum} % For blind text, can be removed after adding actual content

%% Pictures to include in the text can be put in the graphics/ folder
\graphicspath{{graphics/}}

%% For source code highlighting, requires pygments to be installed
%% Compile with the -shell-escape flag!
\usepackage[section]{minted}
%% If you compile with the make_thesis.{bat,sh} script, use the following
%% import instead:
%% \usepackage[section,outputdir=../output]{minted}
\usemintedstyle{solarized-light}
\definecolor{bg}{RGB}{253,246,227} %% Set the background color of the codeframe

%% Change this line to edit the line numbering style:
\renewcommand{\theFancyVerbLine}{\ttfamily\scriptsize\arabic{FancyVerbLine}}

%% Macro definition to load external java source files with \javacode{filename}:
\newmintedfile[javacode]{java}{
    bgcolor=bg,
    fontfamily=tt,
    linenos=true,
    numberblanklines=true,
    numbersep=5pt,
    gobble=0,
    framesep=2mm,
    funcnamehighlighting=true,
    tabsize=4,
    obeytabs=false,
    breaklines=true,
    mathescape=false
    samepage=false,
    showspaces=false,
    showtabs =false,
    texcl=false,
}

% Other packages not already included can be imported here
\usepackage{array}
\usepackage{graphicx}
\usepackage{subcaption} % for subfigures
\usepackage{url}
\def\UrlBreaks{\do\/\do-\do=\do\_}

%%---------- Document metadata -------------------------------------------------
\author{Sarah Eggermont}
\supervisor{Dhr. S. Labijn}
\cosupervisor{Dhr. G. Vande Maele}
\title[Een proof of concept]%
    {Competitief sportplatform met gamification zodat werknemers in een sedentaire job meer bewegen}
\academicyear{\advance\year by -1 \the\year--\advance\year by 1 \the\year}
\examperiod{1}
\degreesought{\IfLanguageName{dutch}{Professionele bachelor in de toegepaste informatica}{Bachelor of applied computer science}}
\partialthesis{false} %% To display 'in partial fulfilment'
%\institution{Internshipcompany BVBA.}

%% Add global exceptions to the hyphenation here
\hyphenation{back-slash}

%% The bibliography (style and settings are  found in hogentthesis.cls)
\addbibresource{bachproef.bib}            %% Bibliography file
\addbibresource{../voorstel/voorstel.bib} %% Bibliography research proposal
\defbibheading{bibempty}{}

%% Prevent empty pages for right-handed chapter starts in twoside mode
\renewcommand{\cleardoublepage}{\clearpage}

\renewcommand{\arraystretch}{1.2}

%% Content starts here.
\begin{document}

%---------- Front matter -------------------------------------------------------

\frontmatter

\hypersetup{pageanchor=false} %% Disable page numbering references
%% Render a Dutch outer title page if the main language is English
\IfLanguageName{english}{%
    %% If necessary, information can be changed here
    \degreesought{Professionele Bachelor toegepaste informatica}%
    \begin{otherlanguage}{dutch}%
       \maketitle%
    \end{otherlanguage}%
}{}

%% Generates title page content
\maketitle
\hypersetup{pageanchor=true}

%%=============================================================================
%% Voorwoord
%%=============================================================================

\chapter*{\IfLanguageName{dutch}{Woord vooraf}{Preface}}%
\label{ch:voorwoord}

%% TODO:
%% Het voorwoord is het enige deel van de bachelorproef waar je vanuit je
%% eigen standpunt (``ik-vorm'') mag schrijven. Je kan hier bv. motiveren
%% waarom jij het onderwerp wil bespreken.
%% Vergeet ook niet te bedanken wie je geholpen/gesteund/... heeft

Met dit onderzoek wil ik voornamelijk onderzoeken hoe je mensen meer aan het bewegen krijgt. Ik ben zelf aan de start van mijn studies rugproblemen beginnen ervaren, en dan voornamelijk tijdens examenperiodes. Na even informeren, bleek dit volledig te wijten aan de vele uren dat ik zat tijdens examenperiodes. Na het bewust inplannen van beweegmomenten merkte ik hier echter verbetering in, en tot op de dag van vandaag probeer ik me hier zeer bewust van te blijven.

Nu ik stage gedaan heb, besef ik echter nog meer hoe moeilijk het kan zijn om voldoende te bewegen wanneer je een zittend beroep hebt. Die reden motiveerde me ook om specifiek te onderzoeken hoe deze doelgroep gemotiveerd en gesensibiliseerd kan worden.

Graag dank ik Dhr. Vande Maele voor zijn expertise binnen het domein, voor de hulp bij het vinden van testpersonen en voor het nalezen van deze bachelorproef. Daarnaast wil ik ook Dhr. Labijn bedanken voor het adviseren omtrent gepaste onderzoekstechnieken en het nalezen van deze scriptie. Door bijdrage van zowel Dhr. Vande Maele als Dhr. Labijn is mijn visie op de mogelijke richtingen waarnaar dit onderzoek kon gaan sterk verbreed. Hierbij hebben zij mij ook begeleid in het afbakenen van dit onderzoek, waarvoor dank.

Verder dank ik graag familie en vrienden voor alle steun tijdens het schrijven, alsook in het bijzonder mijn vriend Stijn Heyman, voor het meermaals doornemen en adviseren van mijn bachelorproef op vlak van vlotheid, samenhang en taal.

Tot slot dank ik u om interesse te tonen en de tijd te nemen dit onderzoek te lezen. Ik hoop dat ik u kan overtuigen om wat regelmatiger recht te staan, de fiets te nemen in plaats van de auto indien mogelijk en regelmatig te bewegen, hoe dit er voor u ook mag uitzien.
%%=============================================================================
%% Samenvatting
%%=============================================================================

% TODO: De "abstract" of samenvatting is een kernachtige (~ 1 blz. voor een thesis) synthese van het document.
%
% Een goede abstract biedt een kernachtig antwoord op volgende vragen:
%
% 1. Waarover gaat de bachelorproef?
% 2. Waarom heb je er over geschreven?
% 3. Hoe heb je het onderzoek uitgevoerd?
% 4. Wat waren de resultaten? Wat blijkt uit je onderzoek?
% 5. Wat betekenen je resultaten? Wat is de relevantie voor het werkveld?
%
% Daarom bestaat een abstract uit volgende componenten:
%
% - inleiding + kaderen thema
% - probleemstelling
% - (centrale) onderzoeksvraag
% - onderzoeksdoelstelling
% - methodologie
% - resultaten (beperk tot de belangrijkste, relevant voor de onderzoeksvraag)
% - conclusies, aanbevelingen, beperkingen



%%---------- Samenvatting -----------------------------------------------------
% De samenvatting in de hoofdtaal van het document

\chapter*{\IfLanguageName{dutch}{Samenvatting}{Abstract}}

Beweging speelt een grote rol in zowel de fysieke als de mentale gezondheid van mensen. Bijna één derde van de wereldbevolking beweegt te weinig en ondervindt hier vroeg of laat de nadelen van. Om die reden bespreekt dit onderzoek hoe een sportplatform gebruik kan maken van gamification om medewerkers van \href{https://en.joule.be/}{Joule}, \href{https://www.ventures4growth.com/en}{Ventures 4 Growth}, \href{https://www.mace-legal.com/}{mace}, \href{https://planetb.life/en}{PlanetB}, \href{https://www.we-are.be/}{we are} en \href{https://www.delaware.pro/en-be}{delaware} aan te zetten om meer te sporten. Bijkomend kan dit er voor zorgen dat de productiviteit positief beïnvloed wordt.

Na een literatuurstudie rond gamification en het belang van beweging, zullen werknemers van eerder genoemde bedrijven geïnterviewd worden om de succescriteria en de benodigdheden van het nieuwe platform te bepalen. Aan de hand van deze criteria zal een platform, in de vorm van een responsive website, gecreëerd worden. Hierin wordt gamification geïmplementeerd en zullen sportgegevens van deelnemende werknemers verzameld en grafisch voorgesteld worden op het platform. Tegelijkertijd zal ook de beleving omtrent het gamification-aspect bevraagd worden. Deze gegevens zullen na een testperiode geanalyseerd worden om te onderzoeken of er een positieve evolutie merkbaar is in de hoeveelheid beweging van de werknemers en of deze ook effectief aan het platform te danken is.

TODO: hier nog effectieve conclusie noteren.

%---------- Inhoud, lijst figuren, ... -----------------------------------------

\tableofcontents

% In a list of figures, the complete caption will be included. To prevent this,
% ALWAYS add a short description in the caption!
%
%  \caption[short description]{elaborate description}
%
% If you do, only the short description will be used in the list of figures

\listoffigures
\listoftables

% If you included tables and/or source code listings, uncomment the appropriate
% lines.
%\listoftables
%\listoflistings

% Als je een lijst van afkortingen of termen wil toevoegen, dan hoort die
% hier thuis. Gebruik bijvoorbeeld de ``glossaries'' package.
% https://www.overleaf.com/learn/latex/Glossaries

%---------- Kern ---------------------------------------------------------------

\mainmatter{}

% De eerste hoofdstukken van een bachelorproef zijn meestal een inleiding op
% het onderwerp, literatuurstudie en verantwoording methodologie.
% Aarzel niet om een meer beschrijvende titel aan deze hoofdstukken te geven of
% om bijvoorbeeld de inleiding en/of stand van zaken over meerdere hoofdstukken
% te verspreiden!

%%=============================================================================
%% Inleiding
%%=============================================================================

\chapter{\IfLanguageName{dutch}{Inleiding}{Introduction}}%
\label{ch:inleiding}

%\begin{itemize}
%  \item context, achtergrond
%  \item afbakenen van het onderwerp
%  \item verantwoording van het onderwerp, methodologie
%  \item probleemstelling
%  \item onderzoeksdoelstelling
%  \item onderzoeksvraag
%  \item \ldots
%\end{itemize}

\section{\IfLanguageName{dutch}{Probleemstelling}{Problem Statement}}%
\label{sec:probleemstelling}

Een sedentaire job wordt geassocieerd met schadelijke gevolgen voor de algemene gezondheid \autocite{Buckley2015}. Het is dus van groot belang dat voldoende beweging een prioriteit is.

\textcite{Hallal2012} beschouwen 31,1\% van de wereldwijde bevolking als inactief. Dit wil zeggen dat, op het moment van dit onderzoek, bijna een derde van de volwassen wereldbevolking de vooropgestelde aanbevelingen van de ``World Health Organization'' (WHO), beschreven door \textcite{Bull2020}, niet haalt. Voor Europa ligt deze waarde zelfs op 34,8\% \autocite{Bull2020}.

Concreet zal deze paper bij (een deel van de) werknemers van \href{https://en.joule.be/}{Joule}, \href{https://www.ventures4growth.com/en}{Ventures 4 Growth}, \href{https://www.mace-legal.com/}{mace}, \href{https://planetb.life/en}{PlanetB}, \href{https://www.we-are.be/}{we are} en \href{https://www.delaware.pro/en-be}{delaware}, die allen een sedentaire job beoefenen, onderzoeken hoe het beweeggedrag is, en of een competitief sportplatform hen kan helpen de vooropgestelde hoeveelheden van de WHO te behalen.

%Uit je probleemstelling moet duidelijk zijn dat je onderzoek een meerwaarde heeft voor een concrete doelgroep. De doelgroep moet goed gedefinieerd en afgelijnd zijn. Doelgroepen als ``bedrijven,'' ``KMO's'', systeembeheerders, enz.~zijn nog te vaag. Als je een lijstje kan maken van de personen/organisaties die een meerwaarde zullen vinden in deze bachelorproef (dit is eigenlijk je steekproefkader), dan is dat een indicatie dat de doelgroep goed gedefinieerd is. Dit kan een enkel bedrijf zijn of zelfs één persoon (je co-promotor/opdrachtgever).

\section{\IfLanguageName{dutch}{Onderzoeksvraag}{Research question}}%
\label{sec:onderzoeksvraag}

%Wees zo concreet mogelijk bij het formuleren van je onderzoeksvraag. Een onderzoeksvraag is trouwens iets waar nog niemand op dit moment een antwoord heeft (voor zover je kan nagaan). Het opzoeken van bestaande informatie (bv. ``welke tools bestaan er voor deze toepassing?'') is dus geen onderzoeksvraag. Je kan de onderzoeksvraag verder specifiëren in deelvragen. Bv.~als je onderzoek gaat over performantiemetingen, dan

Deze paper zal onderzoeken of een competitief sportplatform, dat gebruik maakt van gamification, een positieve invloed kan hebben op het sportgedrag van personen in een sedentaire job. Daarnaast zal het ook achterhalen welke gamificationtechnieken  het meeste succes hebben, en of er technieken zijn die een negatief effect hebben.

\section{\IfLanguageName{dutch}{Onderzoeksdoelstelling}{Research objective}}%
\label{sec:onderzoeksdoelstelling}

Om deze problematiek te proberen verhelpen, zal een sportplatform ontwikkeld worden. Deze ``Proof Of Concept'' (POC) zal het onderzoek faciliteren naar hoe gamification mensen in een sedentaire job kan helpen meer te bewegen in hun dagelijks leven. Gamification is in de literatuur beschreven als het gebruiken van spelelementen die niet aan een spel gerelateerd zijn \autocite{Gaalen2020}. Het gebruik van deze elementen zorgt voor een bepaalde competitie.

%Wat is het beoogde resultaat van je bachelorproef? Wat zijn de criteria voor succes? Beschrijf die zo concreet mogelijk. Gaat het bv.\ om een proof-of-concept, een prototype, een verslag met aanbevelingen, een vergelijkende studie, enz.

\section{\IfLanguageName{dutch}{Opzet van deze bachelorproef}{Structure of this bachelor thesis}}%
\label{sec:opzet-bachelorproef}

De rest van deze bachelorproef is als volgt opgebouwd:

In Hoofdstuk~\ref{ch:stand-van-zaken} wordt een overzicht gegeven van de stand van zaken binnen het onderzoeksdomein, op basis van een literatuurstudie.

In Hoofdstuk~\ref{ch:methodologie} wordt de methodologie toegelicht en worden de gebruikte onderzoekstechnieken besproken om een antwoord te kunnen formuleren op de onderzoeksvragen.

In Hoofdstuk~\ref{ch:proofofconcept} worden de gekozen technologieën van de POC toegelicht en wordt de opbouw van de applicatie besproken.

In Hoofdstuk~\ref{ch:analyse} worden de cijfers geanalyseerd die uit zowel de bevragingen als de website komen.

In Hoofdstuk~\ref{ch:conclusie}, tenslotte, wordt de conclusie gegeven en een antwoord geformuleerd op de onderzoeksvragen. Daarbij wordt ook een aanzet gegeven voor toekomstig onderzoek binnen dit domein.
\chapter{\IfLanguageName{dutch}{Stand van zaken}{State of the art}}%
\label{ch:stand-van-zaken}

% Tip: Begin elk hoofdstuk met een paragraaf inleiding die beschrijft hoe
% dit hoofdstuk past binnen het geheel van de bachelorproef. Geef in het
% bijzonder aan wat de link is met het vorige en volgende hoofdstuk.

% Pas na deze inleidende paragraaf komt de eerste sectiehoofding.

% Dit hoofdstuk bevat je literatuurstudie. De inhoud gaat verder op de inleiding, maar zal het onderwerp van de bachelorproef *diepgaand* uitspitten. De bedoeling is dat de lezer na lezing van dit hoofdstuk helemaal op de hoogte is van de huidige stand van zaken (state-of-the-art) in het onderzoeksdomein. Iemand die niet vertrouwd is met het onderwerp, weet nu voldoende om de rest van het verhaal te kunnen volgen, zonder dat die er nog andere informatie moet over opzoeken \autocite{Pollefliet2011}.

% Je verwijst bij elke bewering die je doet, vakterm die je introduceert, enz.\ naar je bronnen. In \LaTeX{} kan dat met het commando \texttt{$\backslash${textcite\{\}}} of \texttt{$\backslash${autocite\{\}}}. Als argument van het commando geef je de ``sleutel'' van een ``record'' in een bibliografische databank in het Bib\LaTeX{}-formaat (een tekstbestand). Als je expliciet naar de auteur verwijst in de zin (narratieve referentie), gebruik je \texttt{$\backslash${}textcite\{\}}. Soms is de auteursnaam niet expliciet een onderdeel van de zin, dan gebruik je \texttt{$\backslash${}autocite\{\}} (referentie tussen haakjes). Dit gebruik je bv.~bij een citaat, of om in het bijschrift van een overgenomen afbeelding, broncode, tabel, enz. te verwijzen naar de bron. In de volgende paragraaf een voorbeeld van elk.

% \textcite{Knuth1998} schreef een van de standaardwerken over sorteer- en zoekalgoritmen. Experten zijn het erover eens dat cloud computing een interessante opportuniteit vormen, zowel voor gebruikers als voor dienstverleners op vlak van informatietechnologie~\autocite{Creeger2009}.

% Let er ook op: het \texttt{cite}-commando voor de punt, dus binnen de zin. Je verwijst meteen naar een bron in de eerste zin die erop gebaseerd is, dus niet pas op het einde van een paragraaf.

% TODO: dit snap ik echt niet?
Deze paragraaf beschrijft de huidige kennis die er bestaat rond dit onderwerp. Eerst zal het belang van beweging gekaderd worden, waarna een link gelegd zal worden naar de invloed die het heeft op mentale gezondheid en de huidige toestand besproken wordt. Daarna zal gamification zeer uitvoerig geanalyseerd worden, waarna ten slotte een blik zal geworpen worden op bestaande sportapplicaties.

\section{Belang van beweging}

\subsection{Gevolgen van een sedentaire levensstijl}
Bij volwassenen wordt een sedentaire levensstijl geassocieerd met schadelijke gevolgen voor de volgende gezondheidskwesties: sterfte in het algemeen, door hart- en vaatziekten en door kanker. \autocite{Bull2020}. Bovendien wordt het ook gelinkt aan het optreden van hart- en vaatziekten, diabetes type 2 en kanker. \textcite{Stanton2020} koppelen verminderde fysieke activiteit ook aan meer depressie-, angst- en stresssymptomen.

Daarnaast wordt voor mannelijke werknemers van middelbare leeftijd (32 - 69) uit de Verenigde Staten gesteld dat een lage fysieke activiteit op het werk, een significante risicofactor is voor obesitas \autocite{Choi2010}. In Saoedi-Arabië zijn het vooral vrouwen in bureaujobs die drastisch te weinig beweging hebben \autocite{Albawardi2017}. Er kan dus gesteld worden dat er overal ter wereld nood is aan aandacht voor deze problematiek.

Om de kans op gezondheidsproblemen te verkleinen moeten volwassenen, tussen de 18 en 64 jaar oud, volgens de ``World Health Organization'' (WHO) wekelijks 150 à 300 minuten sporten met gemiddelde intensiteit of 75 à 150 minuten met krachtige intensiteit \autocite{Bull2020}. Voor mensen met een beperking worden dezelfde hoeveelheden sport aangeraden, hoewel daar mogelijks samen met een medisch verantwoordelijke bekeken moet worden in welke mate dit mogelijk is, afhankelijk van de beperking. Voor zwangere of net bevallen vrouwen wordt er minstens 150 minuten per week, met gemiddelde intensiteit, aangeraden. In het algemeen kan dus gesteld worden dat voor elk individu, ongeacht de leeftijd, een bepaalde minimum hoeveelheid beweging aangeraden wordt.

\begin{figure}[h]
    \caption[Fysieke inactiviteit bij volwassenen wereldwijd]{Fysieke inactiviteit volwassenen (15+) wereldwijd, bij mannen (A) en vrouwen (B) \autocite{Bull2020}}
    \includegraphics[width=1\textwidth]{Inactiviteit}
    \label{fig:inactivity}
\end{figure}

\textcite{Hallal2012} beschouwen 31,1\% van de wereldwijde bevolking als inactief. Dit wil zeggen dat, op het moment van onderzoek, bijna een derde van de volwassen wereldbevolking de vooropgestelde aanbevelingen van WHO, beschreven door \textcite{Bull2020}, niet haalt. Voor Europa ligt deze waarde zelfs op 34,8\% en zoals op Figuur \ref{fig:inactivity} te zien is, ligt België nog een stuk boven de gemiddelde Europese waarde met 40\% à 49,9\%.

\subsection{Invloed van beweging op mentale gezondheid}
Vele studies hebben aangetoond dat er endogene opioïden aangemaakt worden bij het sporten \autocite{Harber1984}. Endogene opioïden (endorfines, enkefalines en dynorfines) zijn peptiden die biochemische eigenschappen hebben die lijken op opiaten zoals heroïne en morfine. Vooral endorfine als gevolg van training wordt in verband gebracht met zowel fysiologische als psychologische veranderingen \autocite{Dishman2009}.

Deze chemische stoffen komen soms vrij als reactie van het menselijk lichaam op pijn, wat ervoor zorgt dat de pijnperceptie kan veranderen \autocite{Chaudhry2023, Dishman2009}. Daarnaast werden ze ook al geassocieerd met een toestand van plezier \autocite{Chaudhry2023}.

Daarnaast stellen \textcite{Mahindru2023} dat voldoende lichaamsbeweging kan helpen met het verbeteren van slaap, wat op zijn beurt zorgt voor het reguleren van normale hormonale en metabolische processen \autocite{Dolezal2017}. Te weinig slapen heeft zelfs een negatieve impact op de economie: het kost Amerikaanse bedrijven en gezondheidszorginstanties jaarlijks miljarden dollars \autocite{Dolezal2017}.


\section{Beweging en productiviteit}
Wanneer de algemene gezondheid van werknemers slecht is, brengt dit kosten mee voor het bedrijf. \textcite{Sjoegaard2016} beschrijven hoe deze kosten gerelateerd zijn aan de mentale en fysieke afwezigheid van werknemers tijdens het werk, met een verminderde productiviteit tot gevolg. Echter, voor personen die sedentair werk uitvoeren en voornamelijk aan een computer werken, zorgt een verhoogde hoeveelheid sport tijdens de vrije tijd voor minder stress en meer energie op de werkvloer \autocite{Hansen2009}. Daarnaast wordt er voor mensen die in de gezondheidszorg werken, na drie maanden consistent sporten, 8\% productiviteitsstijging waargenomen \autocite{Sjoegaard2016}.

Op die manier leidt het invoeren van regelmatige beweging, door middel van op voorhand opgestelde oefeningen en een zorgvuldige begeleiding, volgens \textcite{Cancelliere2011} tot een positief effect op de productiviteit. In die mate dat \textcite{Sjoegaard2016} stellen dat dit effect de eventuele uitgaven in verband met sportactiviteiten overstijgt.

\section{Gamification}
Volgens \textcite{Deterding2011} is gamification te beschrijven als het gebruiken van speldesignelementen in een niet-spelgerelateerde context.

De laatste jaren wint gamification aan populariteit als manier om gebruikersengagement te ondersteunen en als positieve manieren in het gebruik van diensten te verbeteren \autocite{Hamari2014}.

Gamification bestaat uit drie hoofdonderdelen: de gebruikte techniek, de psychologische uitkomsten en de verdere invloed op het gedrag \autocite{Hamari2014}. Daarnaast zijn sociale aspecten ook essentieel: door het ontstaan van een competitie streven mensen ernaar erkenning te ontvangen \autocite{Hamari2013}.

\subsection{Meest gebruikte technieken}
Volgens \textcite{Legaki2020} kunnen gamificationtechnieken in drie types gecategoriseerd worden: focus op prestaties of uitdagingen, onderdompeling in een verhaal en gebaseerd op sociale interactie.

\subsubsection{Punten en scoreborden}
Volgens \textcite{Hamari2014} zijn punten en scoreborden de meest voorkomende technieken. Punten worden toegekend voor het uitvoeren van vooropgestelde taken, ze focussen dus op prestatie. Aan de hand van deze punten kunnen scoreborden worden opgesteld. Deze scoreborden kunnen de resultaten van meerdere gebruikers tegen elkaar opzetten, wat voor een onderlinge competitie zorgt. Ditzelfde principe kan ook toegepast worden op eigen resultaten, waarbij een gebruiker steeds zichzelf probeert te overtreffen.

\subsubsection{Uitdagingen en badges}
Badges en uitdagingen binnen een spelcontext, tonen veel gelijkenissen met bepaalde marketing tools, zoals klantenkaarten waarop stempels verzameld moeten worden \autocite{Nunes2006}. Dit fenomeen noemen \textcite{Nunes2006} het ``begiftigde vooruitgangseffect'', hierbij zullen mensen meer volharding tonen om een doel te bereiken als ze op een kunstmatige manier vooruitgang kunnen merken richting dat doel.

Een voorbeeld van dit type gamification is \href{https://foursquare.com/}{Foursquare}, deze dienst is gebaseerd op mensen die badges ontgrendelen door bepaalde locaties te bezoeken in de ``echte'' wereld \autocite{Hamari2011}. Maar ook Apple Conditie past dit principe toe met hun badges \autocite{Ha2020}. Op figuur \ref{fig:apple_badges} is zichtbaar hoe deze medailles onderverdeeld worden in meerdere categorieën.

\begin{figure}[h]
    \caption[Badges in de Apple Conditie applicatie]{Een voorbeeld van badges in de Apple Conditie applicatie (Macworld, \href{https://www.macworld.com/article/231140/how-to-get-all-of-the-apple-watch-activity-challenge-badges.html}{2019})}
    \includegraphics[width=1\textwidth]{AppleBadges}
    \label{fig:apple_badges}
\end{figure}

\subsubsection{Levels}
Deze worden gebruikt om gebruikers continu te blijven uitdagen en betrokken te houden \autocite{Dong2012}. Deze techniek is vooral gericht op persoonlijke motivatie en het stimuleren van de eigen vooruitgang en net om die reden is het uitermate geschikt voor het onderwijs \autocite{ManzanoLeon2021}.
 \textcite{ManzanoLeon2021} beschrijven hoe deze techniek voldoet aan de ``Self-Determination Theory'' (SDT), een concept waarin drie psychologische behoeften oorzaak zijn voor intrinsieke motivatie: autonomie, competentie en verbondenheid met anderen.

\subsubsection{Storytelling}
Verhalen kunnen gebruikt worden om onderdompeling en engagement te creëren \autocite{ManzanoLeon2021}. Daarnaast kan er ook de samenhang van een team er mee verbeterd worden: ieder krijgt dat een rol die zijn eigen bijdrage aan het verhaal moet leveren \autocite{ManzanoLeon2021}.

\textcite{Marczewski2015} stelt dat het bij storytelling ook zeer belangrijk is om zowel geloofwaardig te zijn, binnen de regels te blijven van het universum dat gecreëerd wordt, als er voor te zorgen dat elke keuze die gemaakt moet worden, wel degelijk een bijdrage levert aan het geheel. Wanneer een gebruiker namelijk op het einde het gevoel heeft dat diens keuzes nutteloos waren voor het eindresultaat, zal dit vaak resulteren in een teleurstelling in het product \autocite{Marczewski2015}.

Om dit te vermijden, stelt \textcite{Duster1990} dat het beter is om niet noodzakelijke elementen te verwijderen uit een verhaal: ``een geweer dat tijdens het eerste bedrijf van toneelstuk, moet gebruikt worden tegen het derde [bedrijf]''.

\subsubsection{Beloningen}
Daarbij kunnen de gehanteerde technieken verder ingevuld worden naar wens. Zo kunnen beloningen bijvoorbeeld bestaan uit, maar daarom niet gelimiteerd worden tot, een donatie aan een gekozen goed doel of een eervolle vermelding op een intern bedrijfsevenement.

\subsection{Psychologische aspecten}
% TODO: terugkoppelen naar technieken
Gamification is sterk gebaseerd op psychologie. Wanneer psychologische aspecten bevraagd zijn, is er vooral gefocust op motivatie, attitude en plezier \autocite{Hamari2014}. \textcite{Cheong2013} werkte een online quiz uit die gamification gebruikt, met als doel om studeren aan te moedigen door het leuker te maken. Uit een bevraging  na de quiz bleek dat 40,79\% van de deelnemers enthousiast en 46,05\% van de deelnemers tevreden was tijdens de quiz. Daarenboven was het merendeel (77,63\%) van de deelnemers voldoende gemotiveerd om de quiz te vervolledigen.

\subsection{Invloed op gedrag}
\textcite{Kari2016} stellen dat gamification in sportapplicaties een positieve invloed heeft op de intrinsieke bewegingsmotivatie, wat ervoor zorgt dat gebruikers gaan handelen naar een bepaald doelgedrag. \textcite{PoloPena2020} bevestigen dit, maar voegen hier wel aan toe dat gamification een grotere invloed heeft op vrouwen dan op mannen, en op oudere mensen dan op jongere gebruikers. % TODO: WAAROM?

Studies van \textcite{Hamari2013a} hebben aangetoond dat de resultaten van gamification mogelijks niet voor alle gebruikers op lange termijn doeltreffend zijn, en de invoering ervan mogelijks niet op iedereen het gewenste effect heeft.
Anderzijds zal het verwijderen van spelelementen uit een dienst schadelijke gevolgen hebben voor de gebruikers die wel nog betrokken zijn tot het gamification-aspect: zo kan een gebruiker plots al zijn vooruitgang of verdiende badges verliezen.

\subsection{Sociale aspecten}
% TODO: linken met eigen casus
Er zijn twee belangrijke sociale aspecten om in acht te nemen wanneer gamification geïmplementeerd wordt.

Enerzijds is er erkenning, wat beschreven kan worden als de sociale feedback die gebruikers krijgen op hun gedrag \autocite{Cheung2011}.
Wanneer een dienst, zoals het aanbieden van een platform met gamification, erkenning van de andere gebruikers oplevert, wordt die dienst positiever ervaren \autocite{Preece2001}.
\textcite{Hamari2013} suggereren dat er vervolgens, als gevolg van de ontvangen erkenning, een bepaalde bereidheid ontstaat om de erkenning wederkerig te maken. Hierdoor zal de tegenpartij ook op zijn beurt de dienst positiever ervaren.

Anderzijds is er sociale invloed, wat verwijst naar de perceptie van een individu over het belang dat anderen hechten aan een bepaald doelgedrag en of ze verwachten dat iemand dat gedrag zal vertonen \autocite{Ajzen1991}. Specifiek voor een platform dat gamification implementeert, kan er sociale invloed ontstaan door het zien van wat andere gebruikers op het platform presteren. Hierdoor wordt namelijk een verwachtingspatroon gecreëerd en zetten gebruikers elkaar aan tot het behalen van een bepaald doelgedrag, zoals vaker sporten.

\section{Bestaande sportapplicaties}
Hieronder volgt een diverse greep uit de bestaande mobiele en desktop sportapplicaties en -platformen en welke spelelementen daarin gebruikt worden.

\href{https://www.strava.com/}{Strava} is een applicatie waarmee gebruikers hun sportprestaties kunnen bijhouden. \textcite{Barratt2017} stelt dat deze applicatie gamification toepast in de vorm van uitdagingen en persoonlijke trainingsvooruitgang.

\href{https://www.nike.com/be/en/nrc-app}{Nike Run Club} is een loopapplicatie met meer dan 10 miljoen downloads\footnote{\href{https://bootcamp.uxdesign.cc/how-the-nike-run-club-app-got-runners-hooked-2850c7654fc5}{How the Run Club App got runners hooked - Leevey}} op de Google Play Store en de App Store. Deze applicatie maakt het mogelijk voor lopers om een looptraining vast te leggen, punten te verdienen en andere gebruikers uit te dagen. Daarnaast kunnen ze eigen doelstellingen aanmaken en delen, zodat samen naar een gezamenlijk doel gewerkt kan worden \autocite{StaalnackeLarsson2013}.

Een derde voorbeeld is \href{https://connect.garmin.com/}{Garmin Connect}. Het verschil met de vorige twee applicaties, is dat dit platform enkel voor gebruikers met een Garmin-toestel bedoeld is. Deze applicatie focust vooral op badges \autocite{Ilhan2019}. Het behalen van zulke badge kan dan punten opleveren en de mogelijkheid geven tot het behalen van nieuwe, meer uitdagende badges. Om negatieve gevoelens van frustratie op een mindere dag tegen te gaan, krijgt de gebruiker de optie om andere, niet fysieke activiteiten, uit te voeren om ook dan punten te verdienen.

Voor deze casus is echter een platform nodig dat aan de hand van de Strava API een competitie opzet die gericht is op mensen in een zittend beroep. Door de focus op deze doelgroep, zal de gamification hierop afgestemd zijn en zullen mensen minder snel gedemotiveerd zijn. Dergelijke applicatie zal ontwikkeld worden in een volgende fase van het onderzoek.

%%=============================================================================
%% Methodologie
%%=============================================================================

\chapter{\IfLanguageName{dutch}{Methodologie}{Methodology}}%
\label{ch:methodologie}

Dit hoofdstuk schetst in grote lijnen hoe het onderzoek verlopen is en kadert hoe alle elementen van deze bachelorproef samenhangen.

\section{Voorbereidend onderzoek}
Uit de literatuurstudie is reeds gebleken dat er, ondanks het bewezen belang van voldoende beweging, te veel fysieke inactiviteit is bij volwassenen. Voorgaande onderzoeken, zoals onder andere de paper van \textcite{Kari2016}, stellen vast dat de intrinsieke bewegingsmotivatie kan worden gestimuleerd door een sportplatform dat gebruik maakt van gamification. Gezien \textcite{Hamari2013} echter stellen dat dit gewenste effect niet op iedereen van toepassing is, wenst deze bachelorproef te onderzoeken of de eerdere bevindingen kunnen worden bevestigd bij personen met een sedentaire job.

Om deze hypothese te kunnen aftoetsen zal eerst een diverse groep werknemers, die een sedentaire job uitoefenen, worden geselecteerd. Daarna zullen zij worden onderworpen aan een bevraging.

Enerzijds is het de bedoeling te achterhalen in welke mate sport en beweging een rol speelt in hun dagelijkse leven op het moment van het interview. Anderzijds heeft deze ook als doel te achterhalen in welke mate en op welke manieren deze werknemers gestimuleerd kunnen worden om meer te gaan sporten. Dit laat toe om de gamification technieken te identificeren die van toepassing kunnen zijn op deze specifieke casus.

Op basis van al de ontvangen informatie kan bepaalde theorie uit de literatuurstudie reeds worden afgetoetst. Echter vergt de onderzoeksvraag van deze bachelorproef verdergaand onderzoek, zodat op basis van de verkregen inlichtingen de vereisten van de sportapplicatie zijn opgesteld, en de ontwikkeling ervan is begonnen.

\section{Ontwikkeling en ingebruikname van ``Move-it!''}

Steunend op de bekomen resultaten uit zowel de literatuurstudie als de analyse van de gebruikersinterviews, werd een sportapplicatie genaamd ``Move-it!'' ontwikkeld. Het technisch aspect van de ontwikkeling en de geïncorporeerde competitie bevorderende gamification technieken worden verder besproken in hoofdstuk \ref{ch:proofofconcept}.

Deze ``Proof Of Concept'' (POC) dient om de hypotheses te toetsen: heeft het competitief sportplatform, dat gebruik maakt van gamification, een positieve invloed op het sportgedrag van werknemers in een sedentaire job? Welke gamificationtechnieken hebben het meeste succes? Zijn er technieken die een negatief effect hebben?

Om deze vragen te beantwoorden, hebben na een gebruiksperiode van vier weken van "Move-it!" opnieuw gebruikersinterviews plaatsgevonden, met dezelfde personen die voor aanvang van de ontwikkelings- en gebruiksfase geïnterviewd zijn.
Daarbij zijn de vragen quasi identiek gebleven, maar putten de deelnemers voor het beantwoorden van de tweede vragenlijst uit hun ervaringen met het sportplatform. Die ervaringen gingen vooraf door een uitleg over de precieze werking van "Move-it!", waarna wekelijkse herinneringen om de gepresteerde uren in te geven, werden bezorgd. Deze sportgegevens moeten in combinatie met de antwoorden van de vragenlijsten een grondige analyse en daaropvolgende conclusie toelaten, wat in de volgende sectie zal besproken worden.

\section{Analyse van de cijfers}

Zowel de bekomen sportgegevens als de data uit de bevragingen zijn gebruikt om een analyse uit te voeren.

Bij de analyse van de bevragingen wordt er vooral gepeild naar:
\begin{itemize}
    \item de mate van beweging van deelnemers,
    \item wat in de weg staat om meer te bewegen,
    \item welk gevoel gamification teweegbrengt,
    \item wat belangrijk is in een sportapplicatie.
\end{itemize}

In de sportgegevens wordt er vooral gezocht of er een evolutie merkbaar is binnen de tijdspanne van vier weken.

Deze analyse leidt uiteindelijk tot de conclusie van dit onderzoek, die vergeleken wordt met de hypothese die werd opgesteld op basis van eerder onderzoek.

Tenslotte volgt uit deze analyse een discussie over welke aanpassingen aan de POC nog mogelijk zouden zijn en wordt belicht welke mogelijkheden tot toekomstig onderzoek er zijn.

%% TODO: In dit hoofstuk geef je een korte toelichting over hoe je te werk bent
%% gegaan. Verdeel je onderzoek in grote fasen, en licht in elke fase toe wat
%% de doelstelling was, welke deliverables daar uit gekomen zijn, en welke
%% onderzoeksmethoden je daarbij toegepast hebt. Verantwoord waarom je
%% op deze manier te werk gegaan bent.
%%
%% Voorbeelden van zulke fasen zijn: literatuurstudie, opstellen van een
%% requirements-analyse, opstellen long-list (bij vergelijkende studie),
%% selectie van geschikte tools (bij vergelijkende studie, "short-list"),
%% opzetten testopstelling/PoC, uitvoeren testen en verzamelen
%% van resultaten, analyse van resultaten, ...
%%
%% !!!!! LET OP !!!!!
%%
%% Het is uitdrukkelijk NIET de bedoeling dat je het grootste deel van de corpus
%% van je bachelorproef in dit hoofstuk verwerkt! Dit hoofdstuk is eerder een
%% kort overzicht van je plan van aanpak.
%%
%% Maak voor elke fase (behalve het literatuuronderzoek) een NIEUW HOOFDSTUK aan
%% en geef het een gepaste titel.





% Voeg hier je eigen hoofdstukken toe die de ``corpus'' van je bachelorproef
% vormen. De structuur en titels hangen af van je eigen onderzoek. Je kan bv.
% elke fase in je onderzoek in een apart hoofdstuk bespreken.

%\input{...}
%\input{...}
\chapter{\IfLanguageName{dutch}{Proof of concept}{Proof of concept}}%
\label{ch:proofofconcept}

\section{Gekozen technologieën}

\subsection{Framework}

Gezien de expertise van we are, is er voor het platform gekozen voor een responsive React\footnote{\href{https://react.dev/}{https://react.dev/}}-website. Hiervoor is gebruik gemaakt van TypeScript \footnote{\href{https://www.typescriptlang.org/}{https://www.typescriptlang.org/}}.

Voor de implementatie van authenticatie is er gebruik gemaakt van Auth0\footnote{\href{https://auth0.com/}{https://auth0.com/}}. De redenen waarom er voor Auth0 gekozen is, zijn: de mogelijkheid om aan te melden met een Google-account, makkelijke integratie in React en de mogelijkheid tot een gratis abonnement wanneer het aantal gebruikers onder de 7000 blijft, wat voor deze POC meer dan voldoende is.

\subsection{Component libraries}
Gezien de korte ontwikkeltijd die beschikbaar was, zijn Material UI Core\footnote{\href{https://mui.com/core/}{https://mui.com/core/}} en Material UI X\footnote{\href{https://mui.com/x/}{https://mui.com/x/}} component libraries gebruikt voor de ontwikkeling van deze website, om zo het ontwikkelproces te vergemakkelijken. Er is gekozen voor Material UI wegens diens universele look-and-feel, het ruime aanbod aan gratis componenten en de beschikbaarheid van verschillende grafiek-componenten. Daarnaast zijn deze componenten uitermate geschikt voor zowel mobiel als desktop gebruik.

\subsection{Databank}

Deze applicatie gebruikt een graafdatabank, meer specifiek Neo4J\footnote{\href{https://neo4j.com/}{https://neo4j.com/}}. Er is in deze situatie gekozen voor een graaf databank omdat het zeer simpel op te zetten is en er geen databank-schema ontworpen moet worden, wat het dus zeer simpel maakt om snel te ontwikkelen en gaandeweg eventueel zaken aan te passen indien nodig.

Daarnaast zal de analyse van de ingegeven sportdata ook makkelijker verlopen: er zullen geen complexe join-operaties aan te pas komen, wat bij een SQL-databank wel het geval zou zijn.

\subsection{Deployment}

De website wordt gedeployed met behulp van Vercel\footnote{\href{https://vercel.com/home}{https://vercel.com/home}}. De GitHub\footnote{\href{https://github.com/}{https://github.com/}}-repository van deze applicatie is gelinkt aan een Vercel-project, wat er voor zorgt dat de elke nieuwe commit op de main-branch steeds gedeployed zal worden. Dit maakt het zeer eenvoudig om te beheren. Move-it is te vinden op \href{https://move-it-ghent.vercel.app/}{https://move-it-ghent.vercel.app/}.

\section{Opbouw van de applicatie}

De applicatie is opgesteld op basis van de volgende bouwstenen: een centraal dashboard, een overzicht van eigen prestaties, een oplijsting van de teamleden en een profielpagina.

\subsection{Dashboard}
Op het dashboard is de meeste informatie te zien: zowel persoonlijke evolutie over één week en een overzicht van het type uitgevoerde activiteiten (figuur \ref{fig:graphs}) als de team en persoonlijke ranking (figuur \ref{fig:personalRanking} en \ref{fig:teamRanking}) zijn er te vinden. Om kleinere teams evenveel kans te geven, houdt de teamranking rekening met de grote van het team bij de berekening.

\begin{figure}[h]
    \caption[Team ranking]{De team ranking, gezien vanop een laptop.}
    \includegraphics[width=1\textwidth]{TeamRanking}
    \label{fig:teamRanking}
\end{figure}

\begin{figure}[h]
    \caption[Persoonlijke ranking]{De persoonlijke ranking, gezien vanop een laptop.}
    \includegraphics[width=1\textwidth]{PersonalRanking}
    \label{fig:personalRanking}
\end{figure}

Op het dashboard is vooral ingezet op scoreborden en het visualiseren van persoonlijke vooruitgang. De punten zijn hier in de vorm van gepresteerde uren gegeven. Door persoonlijke vooruitgang in een grafiek te tonen, kan een gebruiker gestimuleerd worden om zichzelf steeds te proberen overtreffen.

\begin{figure}[h]
    \caption[Overzicht prestaties dashboard website]{Overzicht van gepresteerde uren de afgelopen week en van het type activiteiten sinds de lancering van de applicatie, gezien vanop een laptop.}
    \includegraphics[width=1\textwidth]{MyGraphs}
    \label{fig:graphs}
\end{figure}

\subsection{Persoonlijk overzicht}
Op het persoonlijk overzicht is een lijst van de gepresteerde activiteiten te zien (figuur \ref{fig:performances} en figuur \ref{fig:performancesMobile}), en kunnen gebruikers ook hun prestaties ingeven. Hier is voldoende ruimte voorzien voor de ingave van technische gegevens, zoals gemiddelde hartslag en hoogtemeters. Afhankelijk van het type activiteit, en wanneer een afstand en duur ingegeven zijn, berekent de applicatie ook een gemiddelde snelheid. Zo kunnen frequente sporters eventueel een vooruitgang opmerken.

\begin{figure}[h]
    \caption[Overzicht activiteiten website]{Overzicht van gepresteerde activiteiten in de applicatie, gezien vanop een laptop.}
    \includegraphics[width=1\textwidth]{MyPerformances}
    \label{fig:performances}
\end{figure}

\begin{figure}[h]
    \caption[Overzicht activiteiten website smartphone]{Overzicht van gepresteerde activiteiten in de applicatie, gezien vanop een smartphone.}
    \includegraphics[width=1\textwidth]{MyPerformancesMobile}
    \label{fig:performancesMobile}
\end{figure}

\subsection{Teamoverzicht}

Dit overzicht is een simpele oplijsting van alle teamleden waartoe de aangemelde gebruiker behoort (figuur \ref{fig:team}).

In de toekomst zou dit uitgebreid kunnen worden met een overzicht van de prestaties van een team, zodat er ook binnen een team gamification speelt en wat mogelijks tot meer motivatie kan leiden.

\begin{figure}[h]
    \caption[Overzicht van teamleden]{Overzicht van teamleden in de applicatie, gezien vanop een laptop. Mailadressen zijn wazig gemaakt om privacy-redenen.}
    \includegraphics[width=1\textwidth]{MyTeam}
    \label{fig:team}
\end{figure}

\subsection{Profiel pagina}

Op deze pagina kan een gebruiker een avatar, team en naam kiezen waarmee die weergegeven wordt.

\chapter{\IfLanguageName{dutch}{Analyse van de cijfers}{Analysis of the figures}}%
\label{ch:analyse}

\section{Resultaten voor gebruik van Move-it!}

Leeftijden bespreken.

Perceptie versus wat aangeraden is -> preventie?

Wat houdt mensen vooral tegen?

Grote voorkeur voor focus op persoonlijke vooruitgang: feature die in Strava betalend is...

Gamification wordt eigenlijk niet zo vaak als vervelend ervaren

Wat wel als vervelend wordt ervaren zijn meldingen en sociale interacties (zoals likes ed)

\section{Sportresultaten}

Evolutie meten. Hoeveel percent haalt de voorgeschreven hoeveelheden.

\section{Resultaten na gebruik van Move-it!}

Nieuwe vragenlijst maken!

- focus op gevoel
- meer sporten in de toekomst?
- wat was anders dan andere sportapplicaties?
    - was dit beter of slechter?
- als u 1 ding zou kunnen veranderen, wat?

%%=============================================================================
%% Conclusie
%%=============================================================================

\chapter{Conclusie}%
\label{ch:conclusie}

% TODO: Trek een duidelijke conclusie, in de vorm van een antwoord op de
% onderzoeksvra(a)g(en). Wat was jouw bijdrage aan het onderzoeksdomein en
% hoe biedt dit meerwaarde aan het vakgebied/doelgroep?
% Reflecteer kritisch over het resultaat. In Engelse teksten wordt deze sectie
% ``Discussion'' genoemd. Had je deze uitkomst verwacht? Zijn er zaken die nog
% niet duidelijk zijn?
% Heeft het onderzoek geleid tot nieuwe vragen die uitnodigen tot verder
%onderzoek?

\section{Bespreking van de resultaten}

- conclusies trekken uit analyse van cijfers, terugkoppelen naar literatuurstudie
- bijdrage aan onderzoeksdomein?

\section{Onduidelijkheden}

- zijn er nog vraagtekens?

\section{Mogelijkheden tot verder onderzoek}

- wat kan nog verder onderzocht worden?


%---------- Bijlagen -----------------------------------------------------------

\appendix

\chapter{Onderzoeksvoorstel}

Het onderwerp van deze bachelorproef is gebaseerd op een onderzoeksvoorstel dat vooraf werd beoordeeld door de promotor. Dat voorstel is opgenomen in deze bijlage.

\section*{Samenvatting}

Beweging speelt een grote rol in zowel de fysieke als de mentale gezondheid van mensen. Bijna één derde van de wereldbevolking beweegt te weinig en ondervindt hier vroeg of laat de nadelen van. Om die reden bespreekt dit onderzoek hoe een sportplatform gebruik kan maken van gamification om medewerkers van \href{https://www.mace-legal.com/}{MACE}, \href{https://www.joule.be/}{Joule} en \href{https://planetb.life/}{PlanetB} aan te zetten om meer te sporten. Bijkomend kan dit er voor zorgen dat de productiviteit positief beïnvloed wordt.

Na een literatuurstudie rond gamification en het belang van beweging, zullen werknemers van eerder genoemde bedrijven geïnterviewd worden om de succescriteria en de benodigdheden van het nieuwe platform te bepalen. Aan de hand van deze criteria zal een platform, in de vorm van een responsive website, gecreëerd worden. Hierin wordt gamification geïmplementeerd en zullen sportgegevens van deelnemende werknemers verzameld en grafisch voorgesteld worden op het platform. Tegelijkertijd zal ook de beleving omtrent het gamification-aspect bevraagd worden. Deze gegevens zullen na een testperiode geanalyseerd worden om te onderzoeken of er een positieve evolutie merkbaar is in de hoeveelheid beweging van de werknemers en of deze ook effectief aan het platform te danken is.

Wanneer rekening gehouden wordt met de sociale aspecten van gamification, geeft de vergaarde kennis momenteel aan dat dit sportplatform wel degelijk een positieve invloed zal hebben op het sportgedrag van de gebruikers. Dit zou op zijn beurt een gunstig effect hebben op de productiviteit van de deelnemende bedrijven.


% Verwijzing naar het bestand met de inhoud van het onderzoeksvoorstel
%---------- Inleiding ---------------------------------------------------------

\section{Introductie}%
\label{sec:introductie}

% TODO: Cijfers over stilzitten, hoeveel een persoon per dag zou moeten staan
Bij volwassenen wordt een sedentaire leefstijl geassocieerd met schadelijke gevolgen voor de volgende gezondheidskwesties: sterfte in het algemeen, sterfte door hart- en vaatziekten en kanker, het voorkomen van hart- en vaatziekten, diabetes type 2 en kanker \autocite{Bull2020}. Het is dus van groot belang dat voldoende beweging een prioriteit is.

Volgens \textcite{Bull2020} moeten volwassenen tussen de 18 en 64 jaar oud, wekelijks 150 à 300 minuten sporten met gemiddelde intensiteit, of 75 à 150 minuten met krachtige intensiteit. Voor mensen met een beperking worden dezelfde hoeveelheden sport aangeraden, hoewel daar mogelijks samen met een medisch verantwoordelijke bekeken moet worden in welke mate dit mogelijk is, afhankelijk van de beperking. Voor zwangere of net bevallen vrouwen wordt er minstens 150 minuten per week, met gemiddelde intensiteit, aangeraden.

Daarnaast gaan verminderde fysieke activiteit ook gepaard met meer depressie-, angst- en stresssymptomen \autocite{Stanton2020}.

Wereldwijd beschouwt \autocite{Hallal2012} 31,1\% van de bevolking als inactief. Dit wil zeggen dat, op het moment van onderzoek, bijna een derde van de volwassen wereldbevolking de vooropgestelde aanbevelingen van ``World Health Organization'' (WHO) niet haalt. Voor Europa ligt deze waarde zelfs op 34,8\%.

% TODO: Verbeteren productiviteit door sporten?
Om deze problematiek te proberen verhelpen, zal een sportplatform ontwikkeld worden. Deze paper onderzoekt hoe gamification hierbij kan helpen. Gamification is in de literatuur beschreven als het gebruiken van spelelementen in een niet-spelgerelateerde context \autocite{Gaalen2020}.


% TODO: beschrijving van literatuurstudie
In de literatuurstudie zal allereerst het concept van gamification uiteengezet worden. Ten tweede worden de populairste technieken, sociale aspecten en de invloed van gamification besproken. Daarna zal het effect dat beweging en een gezonde levensstijl heeft op productiviteit geïllustreerd worden. Ten slotte wordt er gekeken naar bestaande sportapplicaties en -platformen, en beschreven welke vormen van gamification daar in voorkomen.

Nadien zal in de methodologie worden uiteengezet welke stappen er nodig zijn om tot het sportplatform te komen dat, door middel van gamification, werknemers van \textbf{bedrijfX} en \textbf{bedrijfY} motiveert om meer te sporten.

Uiteindelijk kunnen de conclusies van het onderzoek teruggevonden worden.

% Waarover zal je bachelorproef gaan? Introduceer het thema en zorg dat volgende zaken zeker duidelijk aanwezig zijn:

% \begin{itemize}
%    \item kaderen thema
%    \item de doelgroep
%    \item de probleemstelling en (centrale) onderzoeksvraag
%    \item de onderzoeksdoelstelling
% \end{itemize}

% Denk er aan: een typische bachelorproef is \textit{toegepast onderzoek}, wat betekent dat je start vanuit een concrete probleemsituatie in bedrijfscontext, een \textbf{casus}. Het is belangrijk om je onderwerp goed af te bakenen: je gaat voor die \textit{ene specifieke probleemsituatie} op zoek naar een goede oplossing, op basis van de huidige kennis in het vakgebied.

% De doelgroep moet ook concreet en duidelijk zijn, dus geen algemene of vaag gedefinieerde groepen zoals \emph{bedrijven}, \emph{developers}, \emph{Vlamingen}, enz. Je richt je in elk geval op it-professionals, een bachelorproef is geen populariserende tekst. Eén specifiek bedrijf (die te maken hebben met een concrete probleemsituatie) is dus beter dan \emph{bedrijven} in het algemeen.

% Formuleer duidelijk de onderzoeksvraag! De begeleiders lezen nog steeds te veel voorstellen waarin we geen onderzoeksvraag terugvinden.

% Schrijf ook iets over de doelstelling. Wat zie je als het concrete eindresultaat van je onderzoek, naast de uitgeschreven scriptie? Is het een proof-of-concept, een rapport met aanbevelingen, \ldots Met welk eindresultaat kan je je bachelorproef als een succes beschouwen?

%---------- Stand van zaken ---------------------------------------------------

\section{Literatuurstudie}%
\label{sec:state-of-the-art}

% Hier beschrijf je de \emph{state-of-the-art} rondom je gekozen onderzoeksdomein, d.w.z.\ een inleidende, doorlopende tekst over het onderzoeksdomein van je bachelorproef. Je steunt daarbij heel sterk op de professionele \emph{vakliteratuur}, en niet zozeer op populariserende teksten voor een breed publiek. Wat is de huidige stand van zaken in dit domein, en wat zijn nog eventuele open vragen (die misschien de aanleiding waren tot je onderzoeksvraag!)?

\subsection{Gamification}

Volgens \textcite{Deterding2011} is gamification te beschrijven als het gebruiken van speldesignelementen in een niet-spelgerelateerde context. Gamification bestaat uit drie hoofdonderdelen: de gebruikte techniek, de psychologische uitkomsten en de verdere invloed op het gedrag \autocite{Hamari2014}. Daarnaast zijn sociale aspecten ook essentieel: door het ontstaan van een competitie streven mensen ernaar erkenning te ontvangen \autocite{Hamari2013}.

\subsubsection{Populairste technieken}
Volgens \textcite{Hamari2014} zijn punten, scoreborden en vooropgestelde uitdagingen de drie meest voorkomende technieken. Daarnaast komen ook het gebruik van levels \autocite{Dong2012}, beloningen \autocite{Flatla2011} en een overzicht van vooruitgang of het bekomen van badges \autocite{Li2012} veelvuldig voor.

\subsubsection{Sociale aspecten van gamification}
Sociale invloed verwijst naar de perceptie van een individu over het belang dat anderen hechten aan een bepaald doelgedrag en of ze verwachten dat iemand dat gedrag zal vertonen \autocite{Ajzen1991}.

Herkenning beschrijft de sociale feedback die gebruikers krijgen op hun gedrag \autocite{Cheung2011}. \textcite{Hamari2013} suggereren dat het ontvangen van erkenning, een bepaalde bereidheid creëert om anderen binnen eenzelfde dienst wederzijds te erkennen, wat de sociale interactie verder bevordert.

Beide aspecten zijn belangrijk om in acht te nemen wanneer gamification geïmplementeerd wordt. Het is volgens \textcite{Preece2001} namelijk zo dat een service positiever wordt ervaren wanneer het een gevoel van erkenning door andere gebruikers oplevert. Dit zal er op zijn beurt voor zorgen dat de houding van de gebruiker ten opzichte van de dienst positief beïnvloed wordt.

\subsubsection{Invloed van gamification}
Op dit moment is vooral de invloed die gamification heeft op het gedrag van de gebruiker onderzocht. Wanneer psychologische gevolgen ook bevraagd zijn, wordt er vooral gefocust op motivatie, attitude en plezier \autocite{Hamari2014}. Studies van \textcite{Hamari2013a} hebben aangetoond dat de resultaten van gamification mogelijks niet voor alle gebruikers op lange termijn doeltreffend zijn.

\subsection{Beweging en productiviteit}



\subsection{Gamification in bestaande sportapplicaties}


% Zijn er al gelijkaardige onderzoeken gevoerd? Wat concluderen ze? Wat is het verschil met jouw onderzoek?

% Verwijs bij elke introductie van een term of bewering over het domein naar de vakliteratuur, bijvoorbeeld~\autocite{Hykes2013}! Denk zeker goed na welke werken je refereert en waarom \autocite{Bitrian2020}.

% Draag zorg voor correcte literatuurverwijzingen! Een bronvermelding hoort thuis \emph{binnen} de zin waar je je op die bron baseert, dus niet er buiten! Maak meteen een verwijzing als je gebruik maakt van een bron. Doe dit dus \emph{niet} aan het einde van een lange paragraaf. Baseer nooit teveel aansluitende tekst op eenzelfde bron.

% Als je informatie over bronnen verzamelt in JabRef, zorg er dan voor dat alle nodige info aanwezig is om de bron terug te vinden (zoals uitvoerig besproken in de lessen Research Methods).

% Voor literatuurverwijzingen zijn er twee belangrijke commando's:
% \autocite{KEY} => (Auteur, jaartal) Gebruik dit als de naam van de auteur
%   geen onderdeel is van de zin.
% \textcite{KEY} => Auteur (jaartal)  Gebruik dit als de auteursnaam wel een
%   functie heeft in de zin (bv. ``Uit onderzoek door Doll & Hill (1954) bleek
%   ...'')

% Je mag deze sectie nog verder onderverdelen in subsecties als dit de structuur van de tekst kan verduidelijken.

%---------- Methodologie ------------------------------------------------------
\section{Methodologie}%
\label{sec:methodologie}

Het onderzoek zal beginnen met een literatuurstudie over gamification. Hierbij is het belangrijk om technieken te identificeren die van toepassing zijn op deze specifieke casus. De resultaten hiervan kunnen gebruikt worden in de volgende fase.

Deze volgende fase bestaat uit een bevraging bij IT-werknemers van \textbf{enkele} Gentse bedrijven. De bedoeling hiervan is enerzijds achterhalen in welke mate sport en beweging een rol speelt in hun dagelijkse leven op het moment van het interview, en anderzijds te weten komen in welke mate en op welke manieren deze werknemers gestimuleerd kunnen worden om meer te gaan sporten, aan de hand van een competitief sportplatform. Hier kan ook de theorie die verkregen is uit de literatuurstudie getoetst worden.

Steunend op de bekomen resultaten uit zowel de literatuurstudie als de analyse van de gebruikersinterviews, zal dan een applicatie ontwikkeld worden. Het platform zal bestaan in de vorm van een responsive \href{https://react.dev/}{React}-website, geschreven in \href{https://www.typescriptlang.org/}{TypeScript}, waarin sportgegevens van deelnemende werknemers worden verzameld aan de hand van applicaties en hun ''Application Programming Interface'' (API) zoals \href{https://developers.strava.com/}{Strava}, \href{https://dev.fitbit.com/}{Fitbit} en \href{https://developer.garmin.com/gc-developer-program/overview/}{Garmin Connect}. Deze gegevens zullen in combinatie met gamification gebruikt worden om een competitie tussen de deelnemende bedrijven op te zetten.

Deze POC zal dan dienen om de hypotheses te toetsen: heeft het competitief sportplatform, dat gebruik maakt van gamification, een positieve invloed op het sportgedrag van werknemers in een sedentaire job? Om deze vraag te beantwoorden zullen opnieuw gebruikersinterviews plaatsvinden, met dezelfde personen die voor aanvang van de ontwikkelingsfase geïnterviewd zijn en die voor een periode van enkele weken deze POC hebben kunnen gebruiken. Een grondige analyse van de sportgegevens op het platform en de antwoorden van deze ondervraging, leiden tot de conclusie van dit onderzoek.

% Hier beschrijf je hoe je van plan bent het onderzoek te voeren. Welke onderzoekstechniek ga je toepassen om elk van je onderzoeksvragen te beantwoorden? Gebruik je hiervoor literatuurstudie, interviews met belanghebbenden (bv.~voor requirements-analyse), experimenten, simulaties, vergelijkende studie, risico-analyse, PoC, \ldots?

% Uit dit onderdeel moet duidelijk naar voor komen dat je bachelorproef ook technisch voldoen\-de diepgang zal bevatten. Het zou niet kloppen als een bachelorproef informatica ook door bv.\ een student marketing zou kunnen uitgevoerd worden.

% Je beschrijft ook al welke tools (hardware, software, diensten, \ldots) je denkt hiervoor te gebruiken of te ontwikkelen.

% Probeer ook een tijdschatting te maken. Hoe lang zal je met elke fase van je onderzoek bezig zijn en wat zijn de concrete \emph{deliverables} in elke fase?

%---------- Verwachte resultaten ----------------------------------------------
\section{Verwacht resultaat}%
\label{sec:verwachte_resultaten}

De verworven kennis geeft momenteel aan dat een sportplatform met implementatie van gamification wel degelijk een positieve invloed zal hebben op het sportgedrag van gebruikers. Deze conclusie ... onder voorwaarde dat er voldoende aandacht besteed wordt aan het sociale aspect van gamification en de gebruikers ook een persoonlijke ontwikkeling kunnen opmerken op het platform.

% TODO: iets over productiviteit

% Hier beschrijf je welke resultaten je verwacht. Als je metingen en simulaties uitvoert, kan je hier al mock-ups maken van de grafieken samen met de verwachte conclusies. Benoem zeker al je assen en de onderdelen van de grafiek die je gaat gebruiken. Dit zorgt ervoor dat je concreet weet welk soort data je moet verzamelen en hoe je die moet meten.

% Wat heeft de doelgroep van je onderzoek aan het resultaat? Op welke manier zorgt jouw bachelorproef voor een meerwaarde?

% Hier beschrijf je wat je verwacht uit je onderzoek, met de motivatie waarom. Het is \textbf{niet} erg indien uit je onderzoek andere resultaten en conclusies vloeien dan dat je hier beschrijft: het is dan juist interessant om te onderzoeken waarom jouw hypothesen niet overeenkomen met de resultaten.



%%---------- Andere bijlagen --------------------------------------------------
% Voeg hier eventuele andere bijlagen toe. Bv. als je deze BP voor de
% tweede keer indient, een overzicht van de verbeteringen t.o.v. het origineel.
%\input{...}

%%---------- Backmatter, referentielijst ---------------------------------------

\backmatter{}

\setlength\bibitemsep{2pt} %% Add Some space between the bibliograpy entries
\printbibliography[heading=bibintoc]

\end{document}
