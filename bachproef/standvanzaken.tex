\chapter{\IfLanguageName{dutch}{Stand van zaken}{State of the art}}%
\label{ch:stand-van-zaken}

Dit hoofdstuk beschrijft de huidige kennis die er bestaat rond dit onderwerp.

Eerst zal het belang van beweging gekaderd worden, waarna een link gelegd zal worden naar de invloed die het heeft op mentale gezondheid en op productiviteit.

Daarna zal de geschiedenis van gamification besproken worden, waarna het begrip zeer uitvoerig geanalyseerd zal worden. Na de gevolgen van gamification besproken te hebben, zal een blik geworpen worden op de toekomst van gamification. Tenslotte wordt de link tussen gamification en bewegingsmotivatie gekaderd.

Tot slot zal een blik geworpen worden op een greep uit de bestaande sportapplicaties.

\section{Belang van beweging}

\subsection{Fysieke gevolgen van een sedentaire levensstijl}
\label{ssec:fysieke-gevolgen}
Bij volwassenen wordt een sedentaire levensstijl geassocieerd met schadelijke gevolgen voor de volgende gezondheidskwesties: hart- en vaatziekten, diabetes type 2 en kanker, met soms zelfs sterfte tot gevolg \autocite{Bull2020}.

Het valt op dat, naast de ernst van de gezondheidsconsequenties van een sedentaire levensstijl, deze gevolgen zich situeren bij personen van verschillende geslachten, leeftijden en nationaliteiten. Zo wordt voor mannelijke werknemers van middelbare leeftijd (32 - 69) uit de Verenigde Staten gesteld dat een lage fysieke activiteit op het werk, een significante risicofactor is voor obesitas \autocite{Choi2010}. In Saoedi-Arabië is er dan weer voor vrouwen met bureaujobs vastgesteld dat zij veel te weinig beweging hebben, met alle gevolgen van dien \autocite{Albawardi2017}. Er kan om deze redenen gesteld worden dat er overal ter wereld nood is aan aandacht voor deze problematiek.

Om de kans op gezondheidsproblemen te verkleinen, moeten volwassenen, tussen de 18 en 64 jaar oud, volgens de ``World Health Organization'' (WHO) wekelijks minstens 150 à 300 minuten sporten met gemiddelde intensiteit of minstens 75 à 150 minuten met krachtige intensiteit \autocite{Bull2020}. Voor mensen met een beperking worden dezelfde hoeveelheden sport aangeraden, hoewel daar mogelijks samen met een medisch verantwoordelijke bekeken moet worden in welke mate dit mogelijk is, afhankelijk van de beperking. Voor zwangere of net bevallen vrouwen wordt er minstens 150 minuten per week, met gemiddelde intensiteit, aangeraden. In het algemeen kan gesteld worden dat voor elk individu, ongeacht de leeftijd, een bepaalde minimum hoeveelheid beweging aangeraden wordt.

\textcite{Hallal2012} beschouwen 31,1\% van de wereldwijde bevolking als inactief. Dit wil zeggen dat, op het moment van dit onderzoek, bijna een derde van de volwassen wereldbevolking de vooropgestelde aanbevelingen van de WHO, beschreven door \textcite{Bull2020}, niet haalt. Voor Europa ligt deze waarde zelfs op 34,8\% en zoals op figuur \ref{fig:inactivity} te zien is, ligt België nog een stuk boven de gemiddelde Europese waarde met 40\% à 49,9\%.

\begin{figure}[h]
    \caption[Fysieke inactiviteit bij volwassenen wereldwijd]{Fysieke inactiviteit volwassenen (15+) wereldwijd, bij mannen (A) en vrouwen (B) \autocite{Bull2020}.}
    \includegraphics[width=1\textwidth]{Inactiviteit}
    \label{fig:inactivity}
\end{figure}

\subsection{Invloed van beweging op mentale gezondheid}
Naast de fysieke gevolgen van een sedentaire levensstijl, zoals besproken onder subsectie \ref{ssec:fysieke-gevolgen}, zijn er ook mentale gevaren gekoppeld aan een tekort aan beweging. Zo brengen \textcite{Stanton2020} verminderde fysieke activiteit in verband met meer depressie, angst en stresssymptomen.

Bewegen kan deze symptomen tegengaan, er hebben namelijk vele studies aangetoond dat er endogene opioïden aangemaakt worden bij het sporten \autocite{Harber1984}. Deze chemische stoffen komen vrij als reactie van het menselijk lichaam op fysieke en/of mentale pijn, waardoor ze ervoor zorgen dat de pijnperceptie kan veranderen \autocite{Chaudhry2023, Dishman2009}. Daarnaast werden ze ook al geassocieerd met een toestand van plezier \autocite{Chaudhry2023}.
Dat mag niet verbazen eenmaal uiteengezet wordt dat deze endogene opioïden (endorfines, enkefalines en dynorfines) peptiden zijn die biochemische eigenschappen hebben die lijken op opiaten zoals heroïne en morfine. Vooral endorfine als gevolg van training wordt in verband gebracht met zowel fysiologische als psychologische veranderingen, wat een positieve invloed heeft op de mentale staat van een persoon \autocite{Dishman2009}.

Daarnaast stellen \textcite{Mahindru2023} dat voldoende lichaamsbeweging kan helpen met het verbeteren van de slaap. Ook \textcite{Ghrouz2019} stelt dat zowel de kwaliteit als de kwantiteit van de slaap bij adolescenten verbetert na twaalf weken fitnesstraining. Het voorkomen van een slaaptekort, kan op zijn beurt leiden tot het normaal functioneren van het immuunsysteem en het verbeteren van de stemming, het glucosemetabolisme en het cognitieve vermogen \autocite{Mahindru2023}. Daarnaast zorgt voldoende slaap ook voor het reguleren van normale hormonale en metabolische processen en een betere gezondheid in het algemeen \autocite{Dolezal2017}.

\subsection{Invloed van beweging op productiviteit}
Wanneer de algemene gezondheid van werknemers slecht is, brengt dit kosten mee voor het bedrijf. Dit wordt bevestigd door de vaststelling dat te weinig slapen ook een negatieve impact heeft op de economie: het kost Amerikaanse bedrijven en gezondheidszorginstanties jaarlijks miljarden dollars \autocite{Dolezal2017}. \textcite{Sjoegaard2016} beschrijven hoe deze kosten gerelateerd zijn aan de mentale en de fysieke afwezigheid van werknemers tijdens het werk, met een verminderde productiviteit tot gevolg. Voor personen die sedentair werk uitvoeren en voornamelijk aan een computer werken, zorgt een verhoogde hoeveelheid sport tijdens de vrije tijd echter voor minder stress en meer energie op de werkvloer \autocite{Hansen2009}. Daarnaast wordt er voor mensen die in de gezondheidszorg werken, na drie maanden consistent sporten, een productiviteitsstijging van 8\% waargenomen \autocite{Sjoegaard2016}.

Op die manier leidt het invoeren van regelmatige beweging, door middel van op voorhand opgestelde oefeningen en een zorgvuldige begeleiding, volgens \textcite{Cancelliere2011} tot een positief effect op de productiviteit. In die mate dat \textcite{Sjoegaard2016} stellen dat dit effect de eventuele uitgaven in verband met de sportactiviteiten overstijgt, wat ook werkgevers zou moeten motiveren om hun werknemers aan te moedigen om meer te bewegen tijdens hun sedentaire job.

\section{Gamification}

\subsection{Het ontstaan van gamification}
De laatste jaren wint gamification aan populariteit als manier om gebruikersengagement
te ondersteunen en als positieve manier om het gebruik van diensten te
verbeteren \autocite{Hamari2013a}. Echter ligt de initiële oorsprong van gamification, evenwel zonder het zo te benoemen, reeds een eeuw in het verleden. Cracker Jack, een Amerikaans merk uit de voedselindustrie dat onder andere verantwoordelijk is voor de productie van popcorn en nootjes, startte met het integreren van verrassingsspeelgoed in hun verpakkingen. Hiermee wensten zij de consumentenervaring te verbeteren, met als doel de aankoopmotivatie te verhogen \autocite{Khaitova2021}. In navolging daarvan volgden nog vele andere andere bedrijven, tot op de dag van vandaag, dit voorbeeld. Denk bijvoorbeeld aan Ferrero met hun welbekende “kinder surprise”: een chocoladen ei met daarin een stuk speelgoed dat zijn geheimen pas prijsgeeft bij het openmaken.

Gamification heeft sindsdien een enorme evolutie ondergaan. \textcite{Dreimane2021} geeft aan dat Nick Pelling de eerste was om gebruik te maken van gamification zoals het vandaag gekend is. In 2002 ontspon hij zijn consultancy bedrijf “Conundra”, met als doel gamification in consumentengoederen te integreren, door middel van het toevoegen van speelse elementen in hardware. Zijn bedrijf was geen succes en ging failliet, maar desondanks was het idee dat spelelementen buiten een spelcontext konden gebruikt worden, gelanceerd \autocite{Khaitova2021}. Zo ontwikkelde Bunchball in 2007 het eerste gamification platform, waarbij spelmechanismen werden gebruikt in een bedrijfscontext \autocite{Khaitova2021}. Dit zorgt ervoor dat onder andere \textcite{Deterding2011} gamification definitief beschrijven als het gebruiken van speldesignelementen in een niet-spelgerelateerde context.


\subsection{Hedendaagse gamificationtechnieken}
Volgens \textcite{Legaki2020} kunnen gamificationtechnieken in drie types gecategoriseerd worden: er zijn technieken met focus op prestaties of uitdagingen, diegene die zorgen voor de onderdompeling in een verhaal of technieken die gebaseerd zijn op sociale interactie.

\subsubsection{Punten en scoreborden}
\label{sssec:labels_scoreborden}
Volgens \textcite{Hamari2014} zijn de combinatie van punten en scoreborden de meest voorkomende gamificationtechniek. Punten worden toegekend voor het uitvoeren van vooropgestelde taken, wat wil zeggen dat deze techniek ingedeeld kan worden in de eerste categorie van technieken die focussen op prestaties of uitdagingen. Aan de hand van deze punten kunnen de scoreborden worden opgesteld. Deze scoreborden kunnen de resultaten van meerdere gebruikers tegen elkaar opzetten, wat voor een onderlinge competitie zorgt. Ditzelfde principe kan ook toegepast worden op de eigen resultaten, waarbij een gebruiker steeds zichzelf probeert te overtreffen.

\subsubsection{Uitdagingen en badges}
Badges en uitdagingen binnen een spelcontext tonen veel gelijkenissen met bepaalde marketing tools, zoals bijvoorbeeld klantenkaarten waarop stempels verzameld moeten worden \autocite{Nunes2006}. Dit fenomeen noemen \textcite{Nunes2006} het ``Endowed Progress Effect'', of ook wel het ``begiftigde vooruitgangseffect''. Dit effect beschrijft hoe mensen meer volharding tonen om een doel te bereiken wanneer ze op een kunstmatige manier vooruitgang kunnen merken richting dat doel. Zo zullen mensen bijvoorbeeld sneller geneigd zijn een taak uit te voeren met een totaal van tien stappen, waarvan er al twee zijn uitgevoerd, dan dat ze een taak van acht stappen, waar nog niets gedaan is, zullen uitvoeren. Volgens \textcite{Nunes2006} heeft dit meer te maken met de wens om taken te voltooien, dan met de angst om progressie verloren te laten gaan.

Een voorbeeld van dit type gamification is \href{https://foursquare.com/}{Foursquare}, deze dienst is gebaseerd op mensen die badges ontgrendelen door bepaalde locaties te bezoeken in de ``echte'' wereld \autocite{Hamari2011}. Ook Apple Conditie past dit principe toe met hun badges \autocite{Ha2020}. Op figuur \ref{fig:apple_badges} is zichtbaar hoe deze medailles onderverdeeld worden in meerdere categorieën.

\begin{figure}[h]
    \caption[Badges in de Apple Conditie applicatie]{Een voorbeeld van badges in de Apple Conditie applicatie (Macworld, \href{https://www.macworld.com/article/231140/how-to-get-all-of-the-apple-watch-activity-challenge-badges.html}{2019}).}
    \includegraphics[width=1\textwidth]{AppleBadges}
    \label{fig:apple_badges}
\end{figure}

\subsubsection{Levels}
\label{sssec:levels}
Levels worden gebruikt om gebruikers continu te blijven uitdagen en betrokken te houden tot de dienst \autocite{Dong2012}. Deze techniek is vooral gericht op persoonlijke motivatie en het stimuleren van de eigen vooruitgang en net om die reden is het uitermate geschikt voor het onderwijs \autocite{ManzanoLeon2021}.
\textcite{ManzanoLeon2021} beschrijven hoe deze techniek voldoet aan de ``Self-Determination Theory'' (SDT), een concept waarin drie psychologische behoeften de oorzaak zijn voor intrinsieke motivatie van mensen. Deze behoeften zijn autonomie, competentie en verbondenheid met anderen. Gebruikers kunnen, bij het gebruik van levels, namelijk op een autonome manier evolueren doorheen deze levels, ze voelen zich competent wanneer ze slagen en gezien iedereen dezelfde niveaus doorloopt, ontstaat er een gevoel van verbondenheid.

Een bekende applicatie die gebruik maakt van deze gamificationtechniek, is Duolingo\footnote{\href{https://www.duolingo.com/}{https://www.duolingo.com/}}. Wanneer een gebruiker evolueert door de tijd heen, zal die ook stijgen doorheen de levels, wat telkens nieuwe uitdagingen beschikbaar maakt \autocite{Shortt2021}.

\subsubsection{Storytelling}
Verhalen kunnen gebruikt worden om onderdompeling en engagement te creëren \autocite{ManzanoLeon2021}. Daarnaast kan de samenhang van een team er ook mee verbeterd worden: ieder lid van het team krijgt dan een rol die zijn eigen bijdrage aan het verhaal moet leveren \autocite{ManzanoLeon2021}.

\textcite{Marczewski2015} stelt dat het bij storytelling ook zeer belangrijk is om zowel geloofwaardig te zijn, binnen de regels te blijven van het universum dat gecreëerd wordt, als er voor te zorgen dat elke keuze die gemaakt moet worden, wel degelijk een bijdrage levert aan het geheel. Wanneer een gebruiker namelijk op het einde het gevoel heeft dat diens keuzes nutteloos waren voor het eindresultaat, zal dit vaak resulteren in een teleurstelling in het product \autocite{Marczewski2015}.

Om dit te vermijden, stelt \textcite{Duster1990} dat het beter is om niet noodzakelijke elementen te verwijderen uit een verhaal: ``een geweer dat tijdens het eerste bedrijf van een toneelstuk aan de muur gehangen wordt, moet gebruikt worden tegen het derde [bedrijf]''.

Storytelling wordt volgens \textcite{Schmoelz2018} vooral gebruikt binnen het onderwijs, waar het creativiteit bevordert bij het samenwerken aan schrijfopdrachten bijvoorbeeld. Het is om deze redenen minder geschikt voor deze casus.

\subsubsection{Beloningen}
Beloningen kunnen al dan niet in combinatie met één van de hierboven vermeldde technieken gebruikt worden. Ze kunnen bijvoorbeeld bestaan uit, maar daarom niet gelimiteerd worden tot, een donatie aan een gekozen goed doel of een eervolle vermelding op een intern bedrijfsevenement.
Volgens \textcite{Lewis2016} zijn tastbare beloningen echter niet altijd de beste optie, ze kunnen er namelijk voor zorgen dat de ontwikkeling van intrinsieke motivatie afgeremd wordt, terwijl deze net zorgt voor behoud van gedrag op lange termijn. Punten en badges kunnen echter ook als beloning beschouwd worden, en zouden dit effect niet hebben \autocite{Lewis2016}.

\subsection{Werking van gamification}

\subsubsection{Psychologische aspecten}
Gamification is sterk gebaseerd op psychologie. Wanneer in voorgaande onderzoeken de psychologische aspecten van gamification bevraagd zijn, is er vooral gefocust op de motivatie, de attitude en het plezier \autocite{Hamari2014}. \textcite{Cheong2013} werkte bijvoorbeeld een online quiz uit die gamification gebruikt, met als doel om sutdenten aan te moedigen te studeren door het leuker te maken. Uit een bevraging  na de quiz bleek dat 40,79\% van de deelnemers enthousiast en 46,05\% van de deelnemers tevreden was tijdens de quiz. Daarenboven was het merendeel (77,63\%) van de deelnemers voldoende gemotiveerd om de quiz te vervolledigen.

\textbf{Hiërarchie van behoeften}

Een van de oudste en meest bekende motivatietheorieën is afkomstig van psycholoog Abraham Maslow \autocite{Richter2014}. Volgens hem handelen mensen met als doel om hun fysieke en psychologische behoeften te bevredigen. Maslow spreekt over vijf niveaus van noden die menselijke activiteiten aansturen, gaande van fysieke tot persoonlijke behoeften (zie figuur \ref{fig:people-hierarchy}) \autocite{Lilienfeld2014}.

Op basis van dit model, hebben \textcite{Siang2003} een hiërarchie gemaakt die op gamers van toepassing is (zie figuur \ref{fig:gamers-hierarchy}), wat relevant is voor deze casus gezien de gamification component.
Voor deze beide modellen geldt dat de types behoeftes in de onderste rijen van de piramide moeten worden vervuld voor de hogere niveaus in beschouwing kunnen worden genomen \autocite{Richter2014}.
Voor de hiërarchie die op gamers van toepassing is, geldt dat spelers eerst zoeken naar informatie om de regels van het spel te begrijpen. Daarna heeft een speler nood aan veiligheid om door te zetten en mogelijks te kunnen winnen. Ten derde moet een speler het gevoel hebben erbij te horen, waarna die zich goed moet voelen tijdens het spelen van het spel: er moet een gevoel van waardering aanwezig zijn. Op het volgende level willen spelers grotere uitdagingen en willen ze meer begrip verwerven over het spel en diens strategieën. Het voorlaatste niveau gaat over de wens voor bijvoorbeeld goede graphics, bijpassende muziek en geluidseffecten. Ten slotte willen spelers een soort van perfectie bereiken binnen deze virtuele wereld, waarin alles kan en mag, binnen de regels van het spel \autocite{Greitzer2007, Siang2003}.

Bij het implementeren van gamification in een sportplatform is het daarom belangrijk om ervoor te zorgen dat aan de onderste zes niveaus (zie figuur \ref{fig:gamers-hierarchy}) voldaan is, zodat mensen zichzelf ook willen verbeteren via het platform.

\begin{figure}[htbp]
    \begin{minipage}[t]{0.48\linewidth} % adjust width as needed
        \centering
        \caption[Behoeften-hiërarchie algemeen]{De  niveaus van behoeften die menselijke activiteiten aansturen \autocite{Lilienfeld2014}.}
        \includegraphics[width=1\textwidth]{PyramidPeople}
        \label{fig:people-hierarchy}
    \end{minipage}
    \hfill
    \begin{minipage}[t]{0.48\linewidth} % adjust width as needed
        \centering
        \caption[Behoeften-hiërarchie gamers]{De hiërarchie van behoeften van gamers \autocite{Richter2014}.}
        \includegraphics[width=1\textwidth]{PyramidGamers}
        \label{fig:gamers-hierarchy}
    \end{minipage}
\end{figure}

\subsubsection{Invloed op gedrag}
Uit onderzoek van \textcite{Hamari2013a} blijkt dat gamification een invloed kan hebben op een bepaald doelgedrag, maar dat dit niet op alle personen het gewenste effect heeft. Zo hebben onder andere het geslacht en de leeftijd een invloed op de mate waarin er een relatie tussen beiden optreedt.

\textbf{Geslacht}

\textcite{Venkatesh2000} bespreken hoe het proces om een beslissing te maken, verschilt per geslacht. Zo is bijvoorbeeld gevonden dat mannen vaker instrumenteel of proactief gedrag vertonen, wat wilt zeggen dat ze vaker in de aanval of verdediging gaan om zo anderen onder druk te zetten om uiteindelijk op die manier hun doel te bereiken of zin te krijgen \autocite{Spence1980}. Daarnaast stellen \textcite{Hoffman1972, Minton1980} dat ze ook meer taak- en prestatiegericht zijn dan vrouwen.
Er kan verondersteld worden dat zij daardoor vatbaarder zijn voor punten- en scoreborden, gezien de claim van \textcite{Hamari2014} dat dit een prestatiegerichte gamificationtechniek is.

Daartegenover staat dat voor vrouwen de behoefte voor verbondenheid een belangrijkere rol speelt \autocite{Hoffman1972} en ze volgens \textcite{Minton1980, Spence1980} ook meer gericht zijn op sociale relaties. Dit wordt herbevestigd door \textcite{Haferkamp2012} en \textcite{Muscanell2012}, daar zij stellen dat in de huidige online wereld met al diens sociale features, blijkt dat vrouwen de meer sociaal gemotiveerde gebruikers zijn, terwijl mannen eerder focussen op pragmatisch gebruik. Dit zou betekenen dat vrouwen vatbaarder zijn voor sociale invloed van bijvoorbeeld gamification. \textcite{Koivisto2014} besluiten hier daarom over dat het invoeren van sociale features essentieel is om ook vrouwelijke gebruikers te betrekken.

Echter wordt in sommige studies van \textcite{Wang2008} geconcludeerd dat mannen, in bepaalde contexten zoals online leren, wel beïnvloed worden door sociale factoren. Het verschil tussen en mannen en vrouwen lijkt daarom geen kwestie te zijn van wel versus niet onderhevig aan invloeden, maar situeert zich eerder in de mate waarin de vatbaarheid optreedt. \textcite{PoloPena2020} concluderen hierover dat gamification een grotere invloed heeft op vrouwen dan op mannen.

\textbf{Leeftijd}

Wanneer ouderen beslissen over hun intentie om een technologie al dan niet te gebruiken, hechten ze minder belang aan het nut van de technologie dan het jongere deel van de bevolking \autocite{Venkatesh2003}.

Daarnaast ervaren oudere generaties over het algemeen meer computerstress en schatten ze daardoor hun IT-vaardigheden lager in \autocite{Chung2010}. Ze hechten daarom meer belang aan het gebruiksgemak van een systeem. Dit zorgt er, samen met het afgenomen belang van nut, voor dat de afweging tussen het waargenomen gebruikersgemak en het nut belangrijker wordt naarmate een gebruiker ouder wordt, concluderen \textcite{Melenhorst2001}.

Net als vrouwen, worden ouderen ook meer beïnvloed door sociale invloeden. Dit is mogelijk te wijten aan een hogere associatienood \autocite{Morris2000, Venkatesh2000, Wang2008}.

\textcite{Arning2007, Czaja2006} stellen eveneens dat gamification, als die sociale verbondenheid en zelfredzaamheid bevordert, het ook mogelijk maakt om gebruikers van gevorderde leeftijd te betrekken bij diensten.

\textcite{PoloPena2020} concluderen hierdoor dat oudere gebruikers meer onderhevig zijn aan gamification dan jongere gebruikers.

\subsubsection{Sociale aspecten}
\label{ssec:sociale_aspecen}
Er zijn twee belangrijke sociale aspecten om in acht te nemen wanneer gamification geïmplementeerd wordt in een dienst.

Enerzijds is er erkenning, wat beschreven kan worden als de sociale feedback die gebruikers krijgen op hun gedrag \autocite{Cheung2011}.
Wanneer een dienst, zoals het aanbieden van een platform met gamification, erkenning van de andere gebruikers oplevert, wordt die dienst positiever ervaren \autocite{Preece2001}.
\textcite{Hamari2013} suggereren dat er vervolgens, als gevolg van de ontvangen erkenning, een bepaalde bereidheid ontstaat om de erkenning wederkerig te maken. Hierdoor zal de tegenpartij ook op zijn beurt de dienst positiever ervaren.

Anderzijds is er sociale invloed, wat verwijst naar de perceptie van een individu over het belang dat anderen hechten aan een bepaald doelgedrag en of ze verwachten dat iemand dat gedrag zal vertonen \autocite{Ajzen1991}. Specifiek voor een sportplatform dat gamification implementeert, kan er sociale invloed ontstaan door het zien van wat andere gebruikers op het platform presteren. Hierdoor wordt namelijk een verwachtingspatroon gecreëerd en zetten gebruikers elkaar aan tot het behalen van een bepaald doelgedrag, zoals vaker sporten.

Een keerzijde van deze sociale interactie, kan zijn dat het mensen afschrikt. Het kan namelijk zo zijn dat mensen die pas beginnen met sporten zichzelf zullen spiegelen aan anderen die al regelmatig sporten. Hier moet daarom de nodige aandacht aan besteed worden \autocite{Jong2010}.

\subsection{De gevaren van gamification}
\label{ssec:gevaren}

De \textcite{Jong2010} waarschuwden reeds voor de mogelijke keerzijde van sociale interactie als deel van gamification, maar ook daarnaast worden verschillende aandachtspunten rond gamification geopperd door andere onderzoekers. \textcite{Hyrynsalmi2017} beschrijven hoe met “gamification ethiek” een studiedomein is ontstaan dat juist en fout gedrag door of bij gamification oplossingen behelst. Zij dragen hier zelf toe bij door onder de aandacht te brengen dat sportplatformen gebruik maken van bepaalde data, zoals bijvoorbeeld hartslagfrequenties, waarna de gebruiker de controle over deze data verliest. Zeker wanneer data verband houdt met de eigen gezondheid wordt dit door deelnemers als extreem persoonlijk ervaren \autocite{Hyrynsalmi2017}. Dit onderzoek houdt hier rekening mee door expliciet toestemming te vragen voor het gebruik van data, evenals desgewenst bepaalde data te anonimiseren.

\subsection{De toekomst van gamification}

Door het toenemende belang van en aandacht voor gamification geven \autocite{Bezzina2023} aan dat dit sinds 2010 een domein is dat de nodige tractie en bijhorende evolutie genereert. Er is volgens hen binnen de academische wereld belangstelling om gamification met een persoonlijker en aanpasbaard karakter te ontwikkelen, om op die manier via individuele verschillen de intrinsieke motivatie nog meer aan te wakkeren. Artificiële intelligentie (AI) is door zijn configureerbaarheid een uitermate geschikte methode om deze doelen te bereiken, waardoor ook verschillende experts in beide vakgebieden toekomst zien in AI-gestuurde gamification \autocite{Bezzina2023}. AI kan beschreven worden als de simulatie of replicatie van menselijke intelligentie in machines of technologie \autocite{Stewart2020}. Echter blijven de eerder besproken gevaren (zie \ref{ssec:gevaren}) van gamification ook bij deze variant actueel, en komen er potentieel nog andere uitdagingen bij, zoals beschreven door \textcite{Saghiri2022}.

\section{Invloed van gamification op bewegingsmotivatie}

\textcite{Kari2016} stellen dat gamification in sportapplicaties een positieve invloed heeft op de intrinsieke bewegingsmotivatie, wat er voor zorgt dat de gebruikers gaan handelen naar een bepaald doelgedrag, zijnde meer bewegen.

Ook \textcite{Bitrian2020} vindt bevestiging dat een sportapplicatie met gamification autonomie, competentie en verbondenheid creëert, wat volgens \textcite{ManzanoLeon2021} drie psychologische behoeften zijn die kunnen gezien worden als de oorzaak van intrinsieke bewegingsmotivatie (zie ook \ref{sssec:levels}.) Volgens \textcite{Tu2019} kwamen ook \textcite{Lewis2016}, \textcite{Liu2017} en \textcite{Tabak2015} reeds tot bovenstaande conclusies.

Studies van \textcite{Hamari2013a} hebben echter aangetoond dat de resultaten van gamification mogelijk niet voor alle gebruikers op lange termijn doeltreffend zijn, en de invoering ervan mogelijk niet op iedereen het gewenste effect heeft. Daarom wenst dit onderzoek deze lacune deels in te vullen door na te gaan of het gewenste effect wel optreedt bij mensen met een sedentaire job, gezien het grotere gebrek aan beweging tijdens hun job en het daaruit volgende extra grote belang van motiverende factoren.

Daarbij is het belangrijk om rekening te houden met de andere bevinding van \textcite{Hamari2013a}: zo heeft hij aangetoond dat het verwijderen van spelelementen uit een dienst schadelijke gevolgen zal hebben voor de gebruikers die wel nog betrokken zijn bij het gamification-aspect. Op die manier kan een gebruiker plots al zijn vooruitgang of verdiende badges verliezen. Dit kan dan weer eventuele initiële motivatie verstoren.

\section{Bestaande sportapplicaties}
Hieronder volgt een diverse greep uit de bestaande mobiele en desktop sportapplicaties en -platformen en welke spelelementen daarin gebruikt worden.

\href{https://www.strava.com/}{Strava} is een applicatie waarmee gebruikers hun sportprestaties kunnen bijhouden, waar ook een zeer groot aantal features in verwerkt zitten die het de karakteristieken van een sociaal netwerk geeft. Het begon in 2009 als een systeem om outdoor fiets- en hardloopactiviteiten op te nemen, maar is sindsdien uitgebreid met de mogelijkheid om tientallen andere, ook indoor, activiteit types te registreren. \textcite{Barratt2017} stelt dat deze applicatie gamification toepast in de vorm van uitdagingen en persoonlijke trainingsvooruitgang.

Een tweede voorbeeld is de loopapplicatie \href{https://www.nike.com/be/en/nrc-app}{Nike Run Club}, met meer dan 10 miljoen downloads\footnote{\href{https://bootcamp.uxdesign.cc/how-the-nike-run-club-app-got-runners-hooked-2850c7654fc5}{How the Run Club App got runners hooked - Leevey}} in de Google Play Store en de App Store. Deze applicatie maakt het mogelijk voor lopers om een looptraining vast te leggen, punten te verdienen en andere gebruikers uit te dagen. Daarnaast kunnen ze eigen doelstellingen aanmaken en delen, zodat op die manier samen naar een gezamenlijk doel gewerkt kan worden \autocite{StaalnackeLarsson2013}.

Een derde voorbeeld is \href{https://www.runtastic.com/}{Adidas Running}. Deze applicatie is zeer gelijkaardig aan Nike Run Club, met het verschil dat het vanuit een concurrerend merk is. Adidas Running biedt ook op maat gemaakte plannen aan en werkt met een soort van tijdslijn, waarop activiteiten van mensen die de aangemelde gebruiker volgt weergegeven worden. Daarnaast werkt de applicatie ook met een puntensysteem: het behaalde puntenniveau bepaalt het level waarin een gebruiker zich bevindt, wat op zijn beurt toegang geeft tot kortingen of exclusieve deals\footnote{\href{https://www.adidas.com/us/adiclubrewards}{https://www.adidas.com/us/adiclubrewards}}.

Daarnaast bestaat ook \href{https://connect.garmin.com/}{Garmin Connect}. Het verschil met de vorige drie applicaties, is dat dit platform enkel voor gebruikers met een Garmin-toestel bedoeld is. Deze applicatie focust vooral op het behalen van badges \autocite{Ilhan2019}. Het behalen van zulke badge kan dan punten opleveren en de gebruiker de mogelijkheid geven tot het behalen van nieuwe, meer uitdagende badges. Om negatieve gevoelens van frustratie op een mindere dag tegen te gaan, krijgt de gebruiker ook steeds de optie om andere, niet fysieke activiteiten, uit te voeren om ook dan punten te kunnen verdienen, wat zelfs op mindere dagen voor een gevoel van voldoening zorgt.

Een laatste voorbeeld is \href{https://support.apple.com/nl-be/guide/iphone/ipha5dddb411/ios}{Apple Conditie}. Dit platform kan enkel gebruikt worden door gebruikers met een Apple toestel, ENKEL WATCH. Zoals eerder aangehaald, focust deze applicatie op het behalen van medailles, het bijhouden van persoonlijke vooruitgang en indien gewenst ook sociale gamification door het delen van sportieve prestaties binnen de applicatie\footnote{\href{https://support.apple.com/nl-be/guide/iphone/ipha5dddb411/ios}{Aan de slag met Conditie op de iPhone - Apple Support}}.

Voor deze casus is echter een platform nodig dat een competitie opzet die gericht is op mensen met een zittend beroep. Door de focus op deze doelgroep, zullen de gamificationtechnieken hierop afgestemd zijn en zullen mensen minder snel gedemotiveerd zijn. Concreet houdt dit in dat het doel is om boven de minimum aangeraden waarden van de WHO te komen, door middel van elk mogelijk type activiteit. Dergelijke applicatie zal ontwikkeld worden in een volgende fase van het onderzoek.
