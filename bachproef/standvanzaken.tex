\chapter{\IfLanguageName{dutch}{Stand van zaken}{State of the art}}%
\label{ch:stand-van-zaken}

Dit hoofdstuk beschrijft de huidige kennis die er bestaat rond dit onderwerp. Eerst zal het belang van beweging gekaderd worden, waarna een link gelegd zal worden naar de invloed die het heeft op mentale gezondheid en op productiviteit. Daarna zal gamification zeer uitvoerig geanalyseerd worden, waarna ten slotte een blik zal geworpen worden op bestaande sportapplicaties.

\section{Belang van beweging}

\subsection{Gevolgen van een sedentaire levensstijl}
Bij volwassenen wordt een sedentaire levensstijl geassocieerd met schadelijke gevolgen voor de volgende gezondheidskwesties: sterfte in het algemeen, door hart- en vaatziekten en door kanker. \autocite{Bull2020}. Bovendien wordt het ook gelinkt aan het optreden van hart- en vaatziekten, diabetes type 2 en kanker. \textcite{Stanton2020} koppelen verminderde fysieke activiteit ook aan meer depressie, angst en stresssymptomen.

Daarnaast wordt voor mannelijke werknemers van middelbare leeftijd (32 - 69) uit de Verenigde Staten gesteld dat een lage fysieke activiteit op het werk, een significante risicofactor is voor obesitas \autocite{Choi2010}. In Saoedi-Arabië zijn het vooral vrouwen in bureaujobs die drastisch te weinig beweging hebben \autocite{Albawardi2017}. Er kan om deze redenen gesteld worden dat er overal ter wereld nood is aan aandacht voor deze problematiek.

Om de kans op gezondheidsproblemen te verkleinen moeten volwassenen, tussen de 18 en 64 jaar oud, volgens de ``World Health Organization'' (WHO) wekelijks 150 à 300 minuten sporten met gemiddelde intensiteit of 75 à 150 minuten met krachtige intensiteit \autocite{Bull2020}. Voor mensen met een beperking worden dezelfde hoeveelheden sport aangeraden, hoewel daar mogelijks samen met een medisch verantwoordelijke bekeken moet worden in welke mate dit mogelijk is, afhankelijk van de beperking. Voor zwangere of net bevallen vrouwen wordt er minstens 150 minuten per week, met gemiddelde intensiteit, aangeraden. In het algemeen kan gesteld worden dat voor elk individu, ongeacht de leeftijd, een bepaalde minimum hoeveelheid beweging aangeraden wordt.

\textcite{Hallal2012} beschouwen 31,1\% van de wereldwijde bevolking als inactief. Dit wil zeggen dat, op het moment van onderzoek, bijna een derde van de volwassen wereldbevolking de vooropgestelde aanbevelingen van WHO, beschreven door \textcite{Bull2020}, niet haalt. Voor Europa ligt deze waarde zelfs op 34,8\% en zoals op figuur \ref{fig:inactivity} te zien is, ligt België nog een stuk boven de gemiddelde Europese waarde met 40\% à 49,9\%.

\begin{figure}[h]
    \caption[Fysieke inactiviteit bij volwassenen wereldwijd]{Fysieke inactiviteit volwassenen (15+) wereldwijd, bij mannen (A) en vrouwen (B) \autocite{Bull2020}.}
    \includegraphics[width=1\textwidth]{Inactiviteit}
    \label{fig:inactivity}
\end{figure}

\subsection{Invloed van beweging op mentale gezondheid}
Vele studies hebben aangetoond dat er endogene opioïden aangemaakt worden bij het sporten \autocite{Harber1984}. Deze chemische stoffen komen soms ook vrij als reactie van het menselijk lichaam op pijn, waar ze ervoor zorgen dat de pijnperceptie kan veranderen \autocite{Chaudhry2023, Dishman2009}. Daarnaast werden ze ook al geassocieerd met een toestand van plezier \autocite{Chaudhry2023}.
Deze endogene opioïden (endorfines, enkefalines en dynorfines) zijn peptiden die biochemische eigenschappen hebben die lijken op opiaten zoals heroïne en morfine. Vooral endorfine als gevolg van training wordt in verband gebracht met zowel fysiologische als psychologische veranderingen, wat een invloed heeft op de mentale staat van een persoon \autocite{Dishman2009}.

Daarnaast stellen \textcite{Mahindru2023} dat voldoende lichaamsbeweging kan helpen met het verbeteren van slaap, wat op zijn beurt zorgt voor het reguleren van normale hormonale en metabolische processen \autocite{Dolezal2017}. Te weinig slapen heeft zelfs een negatieve impact op de economie: het kost Amerikaanse bedrijven en gezondheidszorginstanties jaarlijks miljarden dollars \autocite{Dolezal2017}.


\subsection{Invloed van beweging op productiviteit}
Wanneer de algemene gezondheid van werknemers slecht is, brengt dit kosten mee voor het bedrijf. \textcite{Sjoegaard2016} beschrijven hoe deze kosten gerelateerd zijn aan de mentale en fysieke afwezigheid van werknemers tijdens het werk, met een verminderde productiviteit tot gevolg. Voor personen die sedentair werk uitvoeren en voornamelijk aan een computer werken, zorgt een verhoogde hoeveelheid sport tijdens de vrije tijd echter voor minder stress en meer energie op de werkvloer \autocite{Hansen2009}. Daarnaast wordt er voor mensen die in de gezondheidszorg werken, na drie maanden consistent sporten, 8\% productiviteitsstijging waargenomen \autocite{Sjoegaard2016}.

Op die manier leidt het invoeren van regelmatige beweging, door middel van op voorhand opgestelde oefeningen en een zorgvuldige begeleiding, volgens \textcite{Cancelliere2011} tot een positief effect op de productiviteit. In die mate dat \textcite{Sjoegaard2016} stellen dat dit effect de eventuele uitgaven in verband met sportactiviteiten overstijgt.

\section{Gamification}
Volgens \textcite{Deterding2011} is gamification te beschrijven als het gebruiken van speldesignelementen in een niet-spelgerelateerde context.

De laatste jaren wint gamification aan populariteit als manier om gebruikersengagement te ondersteunen en als positieve manieren om het gebruik van diensten te verbeteren \autocite{Hamari2014}.

Gamification bestaat uit drie hoofdonderdelen: de gebruikte techniek, de psychologische uitkomsten en de verdere invloed op het gedrag \autocite{Hamari2014}. Daarnaast zijn sociale aspecten ook essentieel: door het ontstaan van een competitie streven mensen ernaar erkenning te ontvangen \autocite{Hamari2013}.

\subsection{Meest gebruikte technieken}
Volgens \textcite{Legaki2020} kunnen gamificationtechnieken in drie types gecategoriseerd worden: focus op prestaties of uitdagingen, onderdompeling in een verhaal of gebaseerd op sociale interactie.

\subsubsection{Punten en scoreborden}
Volgens \textcite{Hamari2014} zijn de combinatie van punten en scoreborden de meest voorkomende techniek. Punten worden toegekend voor het uitvoeren van vooropgestelde taken, ze focussen op prestatie. Aan de hand van deze punten kunnen scoreborden worden opgesteld. Deze scoreborden kunnen de resultaten van meerdere gebruikers tegen elkaar opzetten, wat voor een onderlinge competitie zorgt. Ditzelfde principe kan ook toegepast worden op eigen resultaten, waarbij een gebruiker steeds zichzelf probeert te overtreffen.

\subsubsection{Uitdagingen en badges}
Badges en uitdagingen binnen een spelcontext, tonen veel gelijkenissen met bepaalde marketing tools, zoals klantenkaarten waarop stempels verzameld moeten worden \autocite{Nunes2006}. Dit fenomeen noemen \textcite{Nunes2006} het ``begiftigde vooruitgangseffect'', hierbij tonen mensen meer volharding om een doel te bereiken als ze op een kunstmatige manier vooruitgang kunnen merken richting dat doel.

Een voorbeeld van dit type gamification is \href{https://foursquare.com/}{Foursquare}, deze dienst is gebaseerd op mensen die badges ontgrendelen door bepaalde locaties te bezoeken in de ``echte'' wereld \autocite{Hamari2011}. Maar ook Apple Conditie past dit principe toe met hun badges \autocite{Ha2020}. Op figuur \ref{fig:apple_badges} is zichtbaar hoe deze medailles onderverdeeld worden in meerdere categorieën.

\begin{figure}[h]
    \caption[Badges in de Apple Conditie applicatie]{Een voorbeeld van badges in de Apple Conditie applicatie (Macworld, \href{https://www.macworld.com/article/231140/how-to-get-all-of-the-apple-watch-activity-challenge-badges.html}{2019}).}
    \includegraphics[width=1\textwidth]{AppleBadges}
    \label{fig:apple_badges}
\end{figure}

\subsubsection{Levels}
Levels worden gebruikt om gebruikers continu te blijven uitdagen en betrokken te houden \autocite{Dong2012}. Deze techniek is vooral gericht op persoonlijke motivatie en het stimuleren van de eigen vooruitgang en net om die reden is het uitermate geschikt voor het onderwijs \autocite{ManzanoLeon2021}.
\textcite{ManzanoLeon2021} beschrijven hoe deze techniek voldoet aan de ``Self-Determination Theory'' (SDT), een concept waarin drie psychologische behoeften oorzaak zijn voor intrinsieke motivatie: autonomie, competentie en verbondenheid met anderen. Gebruikers kunnen namelijk op een autonome manier evolueren doorheen deze levels, ze voelen zich competent wanneer ze slagen en gezien iedereen dezelfde niveaus doorloopt, ontstaat er een gevoel van verbondenheid.

\subsubsection{Storytelling}
Verhalen kunnen gebruikt worden om onderdompeling en engagement te creëren \autocite{ManzanoLeon2021}. Daarnaast kan de samenhang van een team er ook mee verbeterd worden: ieder krijgt dan een rol die zijn eigen bijdrage aan het verhaal moet leveren \autocite{ManzanoLeon2021}.

\textcite{Marczewski2015} stelt dat het bij storytelling ook zeer belangrijk is om zowel geloofwaardig te zijn, binnen de regels te blijven van het universum dat gecreëerd wordt, als er voor te zorgen dat elke keuze die gemaakt moet worden, wel degelijk een bijdrage levert aan het geheel. Wanneer een gebruiker namelijk op het einde het gevoel heeft dat diens keuzes nutteloos waren voor het eindresultaat, zal dit vaak resulteren in een teleurstelling in het product \autocite{Marczewski2015}.

Om dit te vermijden, stelt \textcite{Duster1990} dat het beter is om niet noodzakelijke elementen te verwijderen uit een verhaal: ``een geweer dat tijdens het eerste bedrijf van een toneelstuk aan de muur gehangen wordt, moet gebruikt worden tegen het derde [bedrijf]''.

Storytelling is voor deze casus minder van toepassing, het wordt eerder gebruikt in het onderwijs \autocite{Schmoelz2018}.

\subsubsection{Beloningen}
Beloningen kunnen al dan niet in combinatie met één van de hierboven vermeldde technieken gebruikt worden. Ze kunnen bijvoorbeeld bestaan uit, maar daarom niet gelimiteerd worden tot, een donatie aan een gekozen goed doel of een eervolle vermelding op een intern bedrijfsevenement.
Volgens \textcite{Lewis2016} zijn tastbare beloningen echter niet altijd de beste optie, ze kunnen er namelijk voor zorgen dat de ontwikkeling van intrinsieke motivatie afgeremd wordt, terwijl deze net zorgt voor behoud van gedrag op lange termijn. Punten en badges kunnen echter ook als beloning beschouwd worden, en zouden dit effect niet hebben \autocite{Lewis2016}.

\subsection{Psychologische aspecten}
Gamification is sterk gebaseerd op psychologie. Wanneer psychologische aspecten bevraagd zijn, is er vooral gefocust op motivatie, attitude en plezier \autocite{Hamari2014}. \textcite{Cheong2013} werkte een online quiz uit die gamification gebruikt, met als doel om studeren aan te moedigen door het leuker te maken. Uit een bevraging  na de quiz bleek dat 40,79\% van de deelnemers enthousiast en 46,05\% van de deelnemers tevreden was tijdens de quiz. Daarenboven was het merendeel (77,63\%) van de deelnemers voldoende gemotiveerd om de quiz te vervolledigen.

\subsubsection{Hiërarchie van behoeften}
Een van de oudste en meest bekende motivatietheorieën is afkomstig van psycholoog Abraham Maslow \autocite{Richter2014}. Volgens hem handelen mensen met als doel om fysieke en psychologische behoeften te bevredigen. Maslow spreekt over vijf niveaus van noden die menselijke activiteiten aansturen, gaande van fysieke tot persoonlijke behoeften (zie figuur \ref{fig:people-hierarchy}) \autocite{Lilienfeld2014}.

Op basis van dit model, hebben \textcite{Siang2003} een hiërarchie gemaakt die op gamers van toepassing is (zie figuur \ref{fig:gamers-hierarchy}).
Voor beide modellen geldt dat de types behoeftes in de onderste rijen van de piramide moeten worden vervuld voor de hogere niveaus in beschouwing kunnen worden genomen \autocite{Richter2014}.
Eerst zoeken spelers naar informatie om de regels van het spel te begrijpen. Daarna heeft een speler nood aan veiligheid om door te zetten en mogelijks te winnen. Ten derde moet een speler het gevoel hebben erbij te horen, waarna die zich goed moet voelen tijdens het spelen van het spel: er moet een gevoel van waardering aanwezig zijn. Op het volgende level willen spelers grotere uitdagingen en willen ze meer begrijpen over het spel en diens strategieën. Het voorlaatste niveau gaat over de wens voor bijvoorbeeld goede graphics, bijpassende muziek en geluidseffecten. Ten slotte willen spelers een soort van perfectie bereiken binnen deze virtuele wereld, waarin alles kan en mag binnen de regels van het spel \autocite{Greitzer2007, Siang2003}.

Bij het implementeren van gamification in een sportplatform is het daarom belangrijk om ervoor te zorgen dat aan de onderste zes niveaus (zie figuur \ref{fig:gamers-hierarchy}) voldaan is, zodat mensen zichzelf ook willen verbeteren via het platform.

\begin{figure}[htbp]
    \begin{minipage}[t]{0.48\linewidth} % adjust width as needed
        \centering
        \caption[Behoeften-hiërarchie algemeen]{De  niveaus van behoeften die menselijke activiteiten aansturen \autocite{Lilienfeld2014}.}
        \includegraphics[width=1\textwidth]{PyramidPeople}
        \label{fig:people-hierarchy}
    \end{minipage}
    \hfill
    \begin{minipage}[t]{0.48\linewidth} % adjust width as needed
        \centering
        \caption[Behoeften-hiërarchie gamers]{De hiërarchie van behoeften van gamers \autocite{Richter2014}.}
        \includegraphics[width=1\textwidth]{PyramidGamers}
        \label{fig:gamers-hierarchy}
    \end{minipage}
\end{figure}

\subsection{Invloed op gedrag}
\textcite{Kari2016} stellen dat gamification in sportapplicaties een positieve invloed heeft op de intrinsieke bewegingsmotivatie, wat ervoor zorgt dat gebruikers gaan handelen naar een bepaald doelgedrag. \textcite{PoloPena2020} bevestigen dit, maar voegen hier wel aan toe dat gamification een grotere invloed heeft op vrouwen dan op mannen, en op oudere mensen dan op jongere gebruikers.

Studies van \textcite{Hamari2013a} hebben aangetoond dat de resultaten van gamification mogelijks niet voor alle gebruikers op lange termijn doeltreffend zijn, en de invoering ervan mogelijks niet op iedereen het gewenste effect heeft.
Anderzijds zal het verwijderen van spelelementen uit een dienst schadelijke gevolgen hebben voor de gebruikers die wel nog betrokken zijn tot het gamification-aspect: zo kan een gebruiker plots al zijn vooruitgang of verdiende badges verliezen.

\subsubsection{Geslacht}

\textcite{Venkatesh2000} bespreken hoe het proces om een beslissing te maken, verschilt per geslacht. Zo is bijvoorbeeld gevonden dat mannen vaker instrumenteel of proactief gedrag vertonen, wat wilt zeggen dat ze vaker in de aanval of verdediging gaan om zo anderen onder druk te zetten om uiteindelijk op die manier hun doel te bereiken of zin te krijgen \autocite{Spence1980}. Daarnaast stellen \textcite{Hoffman1972, Minton1980} dat ze ook meer taak- en prestatiegericht zijn dan vrouwen. Daartegenover staat dat voor vrouwen de behoefte voor verbondenheid een belangrijkere rol speelt \autocite{Hoffman1972} en ze volgens \textcite{Minton1980, Spence1980} ook meer gericht zijn op sociale relaties. Dit betekent dat vrouwen vatbaarder zijn voor sociale invloed van bijvoorbeeld gamification.

In de huidige online wereld met al diens sociale features, blijkt dat vrouwen de meer sociaal gemotiveerde gebruikers zijn, terwijl mannen eerder focussen op pragmatisch gebruik \autocite{Haferkamp2012, Muscanell2012}. Daartegenover staat dat in sommige studies van \textcite{Wang2008} geconcludeerd wordt dat mannen, in bepaalde contexten zoals online leren, beïnvloed worden door sociale factoren. \textcite{Koivisto2014} besluiten hierover dat het invoeren van sociale features essentieel is als men ook vrouwelijke gebruikers wil betrekken.

\subsubsection{Leeftijd}

Wanneer ouderen beslissen over hun intentie om een technologie al dan niet te gebruiken, hechten ze minder belang aan het nut van de technologie dan jongeren \autocite{Venkatesh2003}. Net als vrouwen, worden ouderen ook meer beïnvloed door sociale invloeden dan jongere gebruikers. Dit is mogelijk te wijten aan een hogere associatienood \autocite{Morris2000, Venkatesh2003, Wang2008}.

Over het algemeen ervaren oudere generaties meer computerstress en schatten ze daardoor hun IT-vaardigheden slechter in \autocite{Chung2010}. Ze hechten ook meer belang aan het gemak in gebruik van een systeem: de afweging tussen het waargenomen gebruiksgemak en het nut en de mogelijke voordelen wordt belangrijker naarmate een gebruiker ouder wordt, concluderen \textcite{Melenhorst2001}.

\textcite{Arning2007, Czaja2006} stellen wel dat gamification, als die sociale verbondenheid en zelfredzaamheid bevordert, het ook mogelijk maakt om gebruikers van gevorderde leeftijd te betrekken bij diensten.

\subsection{Sociale aspecten}
Er zijn twee belangrijke sociale aspecten om in acht te nemen wanneer gamification geïmplementeerd wordt.

Enerzijds is er erkenning, wat beschreven kan worden als de sociale feedback die gebruikers krijgen op hun gedrag \autocite{Cheung2011}.
Wanneer een dienst, zoals het aanbieden van een platform met gamification, erkenning van de andere gebruikers oplevert, wordt die dienst positiever ervaren \autocite{Preece2001}.
\textcite{Hamari2013} suggereren dat er vervolgens, als gevolg van de ontvangen erkenning, een bepaalde bereidheid ontstaat om de erkenning wederkerig te maken. Hierdoor zal de tegenpartij ook op zijn beurt de dienst positiever ervaren.

Anderzijds is er sociale invloed, wat verwijst naar de perceptie van een individu over het belang dat anderen hechten aan een bepaald doelgedrag en of ze verwachten dat iemand dat gedrag zal vertonen \autocite{Ajzen1991}. Specifiek voor een platform dat gamification implementeert, kan er sociale invloed ontstaan door het zien van wat andere gebruikers op het platform presteren. Hierdoor wordt namelijk een verwachtingspatroon gecreëerd en zetten gebruikers elkaar aan tot het behalen van een bepaald doelgedrag, zoals vaker sporten.

Een keerzijde van deze sociale interactie, kan zijn dat het mensen afschrikt. Het kan namelijk zo zijn dat mensen die pas beginnen met sporten zichzelf zullen spiegelen aan anderen die al regelmatig sporten. Hier moet de nodige aandacht aan besteed worden.

\section{Bestaande sportapplicaties}
Hieronder volgt een diverse greep uit de bestaande mobiele en desktop sportapplicaties en -platformen en welke spelelementen daarin gebruikt worden.

\href{https://www.strava.com/}{Strava} is een applicatie waarmee gebruikers hun sportprestaties kunnen bijhouden. \textcite{Barratt2017} stelt dat deze applicatie gamification toepast in de vorm van uitdagingen en persoonlijke trainingsvooruitgang.

Een tweede voorbeeld is de loopapplicatie \href{https://www.nike.com/be/en/nrc-app}{Nike Run Club}, met meer dan 10 miljoen downloads\footnote{\href{https://bootcamp.uxdesign.cc/how-the-nike-run-club-app-got-runners-hooked-2850c7654fc5}{How the Run Club App got runners hooked - Leevey}} op de Google Play Store en de App Store. Deze applicatie maakt het mogelijk voor lopers om een looptraining vast te leggen, punten te verdienen en andere gebruikers uit te dagen. Daarnaast kunnen ze eigen doelstellingen aanmaken en delen, zodat samen naar een gezamenlijk doel gewerkt kan worden \autocite{StaalnackeLarsson2013}.

Daarnaast bestaat ook \href{https://connect.garmin.com/}{Garmin Connect}. Het verschil met de vorige twee applicaties, is dat dit platform enkel voor gebruikers met een Garmin-toestel bedoeld is. Deze applicatie focust vooral op badges \autocite{Ilhan2019}. Het behalen van zulke badge kan dan punten opleveren en de mogelijkheid geven tot het behalen van nieuwe, meer uitdagende badges. Om negatieve gevoelens van frustratie op een mindere dag tegen te gaan, krijgt de gebruiker de optie om andere, niet fysieke activiteiten, uit te voeren om ook dan punten te verdienen.

Een laatste voorbeeld is \href{https://support.apple.com/nl-be/guide/iphone/ipha5dddb411/ios}{Apple Conditie}. Dit platform kan enkel gebruikt worden door gebruikers met een Apple toestel. Zoals eerder aangehaald, focust deze applicatie op het behalen van medailles, het bijhouden van persoonlijke vooruitgang en indien gewenst ook sociale gamification door het delen van sportieve prestaties binnen de applicatie.

Voor deze casus is echter een platform nodig dat een competitie opzet die gericht is op mensen in een zittend beroep. Door de focus op deze doelgroep, zal de gamification hierop afgestemd zijn en zullen mensen minder snel gedemotiveerd zijn. Concreet houdt dit in dat het enige doel is om boven de minimum aangeraden waarden van de WHO te komen, door middel van elk type activiteit dat men kan of wil beoefenen. Dergelijke applicatie zal ontwikkeld worden in een volgende fase van het onderzoek.
