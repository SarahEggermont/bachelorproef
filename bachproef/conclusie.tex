%%=============================================================================
%% Conclusie
%%=============================================================================

\chapter{Conclusie}%
\label{ch:conclusie}

Op basis van de bekomen resultaten uit de analyse, zal dit hoofdstuk de conclusies formuleren van dit onderzoek.

% TODO: Trek een duidelijke conclusie, in de vorm van een antwoord op de
% onderzoeksvra(a)g(en). Wat was jouw bijdrage aan het onderzoeksdomein en
% hoe biedt dit meerwaarde aan het vakgebied/doelgroep?
% Reflecteer kritisch over het resultaat. In Engelse teksten wordt deze sectie
% ``Discussion'' genoemd. Had je deze uitkomst verwacht? Zijn er zaken die nog
% niet duidelijk zijn?
% Heeft het onderzoek geleid tot nieuwe vragen die uitnodigen tot verder
%onderzoek?

\section{Bespreking van de resultaten}

Uit de eerste vragenlijst kan geconcludeerd worden dat vooral de hoeveelheid dagelijkse beweging met gemiddelde intensiteit niet behaald wordt. Een mogelijke oorzaak van dit resultaat, kan zijn dat mensen die al veel sporten, enkel intensieve activiteiten beschouwen als beweging.

Hier komt nog een conclusie van de waargenomen sportactiviteiten, waarna er teruggekoppeld zal worden naar de literatuurstudie.

Hier zal ook besproken worden of mogelijks de piramide van behoeften (eerder besproken in de literatuurstudie) invloed heeft op het resultaat van dit onderzoek, met andere woorden: werden mensen niet uitgedaagd door het sportplatform doordat er aan de noden in de onderste lagen van de piramide niet voldaan werd.

Antwoord op onderzoeksvraag: ja, een competitief sportplatform kan mensen in een sedentaire job helpen om meer te bewegen, maar enkel als
er voldoende rekening gehouden wordt met hoe gamification geïmplementeerd wordt,
het makkelijk gemaakt wordt op sportgegevens in te geven en er voldoende mensen meedoen die elkaar kennen.

\section{Mogelijkheden tot verder onderzoek}

Zoals eerder besproken, leiden de volgende onduidelijkheden of vragen tot verder onderzoek:

\begin{itemize}
    \item Hoe kan er een onderscheid gemaakt worden tussen intensieve activiteiten en activiteiten met gemiddelde intensiteit?
    \item Geven personen die weinig maar wel intensief sporten enkel deze gegevens in, waardoor het lijkt dat ze minder doen dan mensen die regelmatig een beetje bewegen? Of bewegen ze effectief weinig naast deze intensieve activiteiten?
    \item Welke conclusies kunnen getrokken worden wanneer er een grotere steekproef genomen wordt?
    \item Welk effect heeft een sportplatform met gamification er uit voor mensen die ouder zijn dan de deelnemers van dit onderzoek?
    \item Hoe willen mensen gemotiveerd worden als ze geen meldingen willen ontvangen?
\end{itemize}