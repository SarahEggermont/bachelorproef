%%=============================================================================
%% Conclusie
%%=============================================================================

\chapter{Conclusie}%
\label{ch:conclusie}

% TODO: Trek een duidelijke conclusie, in de vorm van een antwoord op de
% onderzoeksvra(a)g(en). Wat was jouw bijdrage aan het onderzoeksdomein en
% hoe biedt dit meerwaarde aan het vakgebied/doelgroep?
% Reflecteer kritisch over het resultaat. In Engelse teksten wordt deze sectie
% ``Discussion'' genoemd. Had je deze uitkomst verwacht? Zijn er zaken die nog
% niet duidelijk zijn?
% Heeft het onderzoek geleid tot nieuwe vragen die uitnodigen tot verder
%onderzoek?

\section{Bespreking van de resultaten}

Uit de cijfers kan geconcludeerd worden dat vooral de hoeveelheid dagelijkse beweging met gemiddelde intensiteit niet behaald wordt. Een mogelijke oorzaak van dit resultaat, kan zijn dat mensen die al veel sporten, enkel intensieve activiteiten registreren.

- conclusies trekken uit analyse van cijfers, terugkoppelen naar literatuurstudie
- bijdrage aan onderzoeksdomein?

- teruggrijpen naar piramide van behoeften

\section{Onduidelijkheden}

- zijn er nog vraagtekens?

\section{Mogelijkheden tot verder onderzoek}

- wat kan nog verder onderzocht worden?
