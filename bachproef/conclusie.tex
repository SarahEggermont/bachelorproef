%%=============================================================================
%% Conclusie
%%=============================================================================

\chapter{Conclusie}%
\label{ch:conclusie}

Op basis van de bekomen resultaten uit de analyse, zal dit hoofdstuk de conclusies formuleren van dit onderzoek.

% Trek een duidelijke conclusie, in de vorm van een antwoord op de
% onderzoeksvra(a)g(en). Wat was jouw bijdrage aan het onderzoeksdomein en
% hoe biedt dit meerwaarde aan het vakgebied/doelgroep?
% Reflecteer kritisch over het resultaat. In Engelse teksten wordt deze sectie
% ``Discussion'' genoemd. Had je deze uitkomst verwacht? Zijn er zaken die nog
% niet duidelijk zijn?
% Heeft het onderzoek geleid tot nieuwe vragen die uitnodigen tot verder
%onderzoek?

\section{Conclusies van het onderzoek}

Deze bachelorproef trachtte te onderzoeken of een sportplatform, dat gebruik \linebreak maakt van gamification, kan bijdragen aan een actievere levensstijl voor mensen met een sedentaire job.
Eerst en vooral werden het belang van beweging, gamification en de invloed die gamification kan hebben op beweging, besproken. Daarna werden de antwoorden uit de eerste vragenlijst geanalyseerd waarna de data die uit ``Move-it!'' kwam, in kaart werd gebracht. Tenslotte werd deze data in combinatie met de antwoorden uit een tweede vragenlijst bestudeerd, om zo tot de conclusies van dit onderzoek te komen.

Deze paper suggereert dat een competitief sportplatform, dat gebruik maakt van gamification, een positieve invloed heeft op het sportgedrag van werknemers met een sedentaire job. Dit wil zeggen dat de resultaten van \textcite{Kari2016}, \textcite{Tu2019}, \textcite{Lewis2016}, \textcite{Liu2017} en \textcite{Tabak2015} uitgebreid kunnen worden naar mensen die een zittend beroep uitoefenen. Daarenboven stelt dit ook voor dat een competitief sportplatform ervoor kan zorgen dat personen met een sedentaire job het tekort aan beweging in hun vrije tijd wel kunnen compenseren, wat een aanvulling zou kunnen zijn op het onderzoek van \textcite{Vandelanotte2015}.

Op vlak van gamification, hebben punten en scoreborden het meeste succes, vooral wanneer dit zorgt voor een onderlinge competitie. Er zijn in dit onderzoek geen gamification technieken naar boven gekomen dit een negatief effect hebben op de hoeveelheid beweging. Specifiek voor sportapplicaties, suggereert dit onderzoek echter wel dat gebruikers het storend vinden om meldingen te ontvangen als deel van de gamification. In welke mate dit een mogelijke storende factor is in het aanwakkeren van intrinsieke motivatie, moet verder onderzoek uitwijzen.

\section{Suggesties voor verder onderzoek}

Gezien het beperkte aantal deelnemers, moeten de bekomen resultaten en conclusies met voorzichtigheid benaderd worden en kunnen ze niet veralgemeend worden voor alle personen met een sedentaire job.

Een eerste aanbeveling voor verder onderzoek, is dan ook het vergroten van de steekproef. Daarnaast leiden ook de volgende onduidelijkheden of vragen tot verder onderzoek:

\begin{itemize}
    \item Zijn de resultaten verschillend wanneer er een onderscheid gemaakt kan worden tussen intensieve activiteiten en activiteiten met gemiddelde intensiteit?
    \item Geven personen die weinig maar wel intensief sporten enkel deze gegevens in, waardoor het lijkt dat ze minder doen dan mensen die regelmatig minder intensief bewegen? Of bewegen ze effectief weinig naast deze intensieve activiteiten?
    \item Welk effect heeft een sportplatform met gamification voor mensen die ouder zijn dan de deelnemers van dit onderzoek?
    \item Hoe willen mensen gemotiveerd worden als ze geen meldingen willen ontvangen?
\end{itemize}