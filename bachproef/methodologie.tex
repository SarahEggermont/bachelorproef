%%=============================================================================
%% Methodologie
%%=============================================================================

\chapter{\IfLanguageName{dutch}{Methodologie}{Methodology}}%
\label{ch:methodologie}

Dit hoofdstuk schetst in grote lijnen hoe het onderzoek verlopen is en kadert hoe alle elementen van deze bachelorproef samenhangen.

\section{Voorbereidend onderzoek}
Uit de literatuurstudie is reeds gebleken dat er, ondanks het bewezen belang van voldoende beweging, te veel fysieke inactiviteit is bij volwassenen. Voorgaande onderzoeken, zoals onder andere de paper van \textcite{Kari2016}, stellen vast dat de intrinsieke bewegingsmotivatie kan worden gestimuleerd door een sportplatform dat gebruik maakt van gamification. Gezien \textcite{Hamari2013} echter stellen dat dit gewenste effect niet op iedereen van toepassing is, wenst deze bachelorproef te onderzoeken of de eerdere bevindingen kunnen worden bevestigd bij personen met een sedentaire job.

Om dit uit te voeren, zal eerst een diverse groep werknemers, die een sedentaire job uitoefenen, worden geselecteerd. Daarna zullen zij worden onderworpen aan een eerste bevraging.

Enerzijds is het de bedoeling te achterhalen in welke mate sport en beweging een rol speelt in hun dagelijkse leven op het moment van het interview. Anderzijds heeft deze ook als doel te achterhalen in welke mate en op welke manieren deze werknemers gestimuleerd kunnen worden om meer te gaan sporten. Dit laat toe om de gamificationtechnieken te identificeren die van toepassing kunnen zijn op deze specifieke casus.

Op basis van al de ontvangen informatie kan bepaalde theorie uit de literatuurstudie reeds worden afgetoetst. Echter vergt de onderzoeksvraag van deze bachelorproef verdergaand onderzoek, zodat op basis van de verkregen inlichtingen de vereisten van de sportapplicatie zijn opgesteld, en de ontwikkeling ervan is begonnen.

\section{Ontwikkeling en ingebruikname van ``Move-it!''}

Steunend op de bekomen resultaten uit zowel de literatuurstudie als de analyse van de gebruikersinterviews, werd een sportapplicatie genaamd ``Move-it!'' ontwikkeld. Het technisch aspect van de ontwikkeling en de geïncorporeerde competitie bevorderende gamificationtechnieken worden verder besproken in hoofdstuk \ref{ch:proofofconcept}.

Deze ``Proof Of Concept'' (POC) dient om een antwoord te bieden op de onderzoeksvragen: heeft het competitief sportplatform, dat gebruik maakt van gamification, een positieve invloed op het sportgedrag van werknemers in een sedentaire job? Welke gamificationtechnieken hebben het meeste succes? Zijn er technieken die een negatief effect hebben?

Om deze vragen te beantwoorden, hebben na een gebruiksperiode van vier weken van "Move-it!" gebruikersinterviews plaatsgevonden, met dezelfde personen die voor aanvang van de ontwikkelings- en gebruiksfase geïnterviewd zijn.
Daarbij zijn de vragen quasi identiek gebleven, maar putten de deelnemers voor het beantwoorden van de tweede vragenlijst uit hun ervaringen met het sportplatform. Die ervaringen gingen vooraf door een uitleg over de precieze werking van "Move-it!", waardoor aan de eerste behoefte, zoals beschreven door \textcite{Siang2003} (zie \ref{ssec:werking-gamification}), werd voldaan. De vastgelegde hiërarchie der behoeften in een spelcontext stelde namelijk dat er eerst nood is aan regels om het spel te begrijpen. Daarna werden wekelijkse herinneringen om de gepresteerde uren in te geven, bezorgd. Deze sportgegevens moeten in combinatie met de antwoorden van de vragenlijsten een grondige analyse en daaropvolgende conclusie toelaten, wat in de volgende sectie besproken zal worden.

\section{Analyse van de cijfers}

Zowel de bekomen sportgegevens als de data uit de bevragingen zijn gebruikt om een analyse uit te voeren.

Bij de analyse van de bevragingen wordt er vooral gepeild naar:
\begin{itemize}
    \item de mate van beweging van deelnemers,
    \item wat in de weg staat om meer te bewegen,
    \item welk gevoel gamification teweegbrengt,
    \item wat belangrijk is in een sportapplicatie.
\end{itemize}

In de sportgegevens wordt er vooral gezocht of er een evolutie merkbaar is binnen de tijdspanne van vier weken.

Deze analyse leidt uiteindelijk tot de conclusie van dit onderzoek, die vergeleken wordt met de hypothese die werd opgesteld op basis van eerder onderzoek.

Tenslotte volgt uit deze analyse ook een overzicht van welke mogelijkheden tot toekomstig onderzoek er zijn.

%% TODO: In dit hoofstuk geef je een korte toelichting over hoe je te werk bent
%% gegaan. Verdeel je onderzoek in grote fasen, en licht in elke fase toe wat
%% de doelstelling was, welke deliverables daar uit gekomen zijn, en welke
%% onderzoeksmethoden je daarbij toegepast hebt. Verantwoord waarom je
%% op deze manier te werk gegaan bent.
%%
%% Voorbeelden van zulke fasen zijn: literatuurstudie, opstellen van een
%% requirements-analyse, opstellen long-list (bij vergelijkende studie),
%% selectie van geschikte tools (bij vergelijkende studie, "short-list"),
%% opzetten testopstelling/PoC, uitvoeren testen en verzamelen
%% van resultaten, analyse van resultaten, ...
%%
%% !!!!! LET OP !!!!!
%%
%% Het is uitdrukkelijk NIET de bedoeling dat je het grootste deel van de corpus
%% van je bachelorproef in dit hoofstuk verwerkt! Dit hoofdstuk is eerder een
%% kort overzicht van je plan van aanpak.
%%
%% Maak voor elke fase (behalve het literatuuronderzoek) een NIEUW HOOFDSTUK aan
%% en geef het een gepaste titel.



