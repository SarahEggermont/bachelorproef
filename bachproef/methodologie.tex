%%=============================================================================
%% Methodologie
%%=============================================================================

\chapter{\IfLanguageName{dutch}{Methodologie}{Methodology}}%
\label{ch:methodologie}

\section{Voorbereidend onderzoek}
Deze bachelorproef zal beginnen met een literatuurstudie die het belang van voldoende beweging kadert, vervolgens bespreekt wat het effect hiervan is op productiviteit en tenslotte gamification verder uitdiept. Het zal bij dit laatste belangrijk zijn om technieken te identificeren die van toepassing zijn op deze specifieke casus. De resultaten hiervan kunnen gebruikt worden in de volgende fase.

Deze volgende fase zal bestaan uit een online bevraging bij werknemers van verschillende bedrijven. De bedoeling hiervan is enerzijds achterhalen in welke mate sport en beweging een rol speelt in hun dagelijkse leven op het moment van het interview. Anderzijds is het ook de bedoeling te weten komen in welke mate en op welke manieren deze werknemers gestimuleerd kunnen worden om meer te gaan sporten, aan de hand van een competitief sportplatform. Hier zal ook de theorie die verkregen is uit de literatuurstudie getoetst kunnen worden.

Op basis van deze informatie zullen de vereisten van de sportapplicatie opgesteld kunnen worden, en kan de ontwikkeling ervan beginnen.

\section{Ontwikkeling ``Move-it!''}

Steunend op de bekomen resultaten uit zowel de literatuurstudie als de analyse van de gebruikersinterviews, zal een een applicatie ontwikkeld worden.

Deze ``Proof Of Concept'' (POC) zal dienen om de hypotheses te toetsen: heeft het competitief sportplatform, dat gebruik maakt van gamification, een positieve invloed op het sportgedrag van werknemers in een sedentaire job? Welke gamificationtechnieken hebben het meeste succes? Zijn er technieken die een negatief effect hebben?

Om deze vragen te beantwoorden, zullen opnieuw gebruikersinterviews plaatsvinden, met dezelfde personen die voor aanvang van de ontwikkelingsfase geïnterviewd zijn en die voor een periode van enkele weken deze POC hebben kunnen gebruiken.

\section{Ingebruikname van het sportplatform}

Gedurende vier weken zullen deelnemers dit sportplatform kunnen gebruiken. Wekelijks werden ze er aan herinnerd om hun gepresteerde uren in te geven, voor het geval ze dit vergeten zouden zijn. Deze sportgegevens zullen in combinatie met gamification gebruikt worden om een competitie tussen de deelnemers op te zetten.

Daarnaast zullen de psychologische en sociale belevingen systematisch bevraagd worden bij het gebruik van het platform, om eventuele evoluties hiervan te kunnen opmerken. Een grondige analyse van de bekomen sportgegevens op het platform, de antwoorden van de interviews en de resultaten van deze tussentijdse ondervragingen, leiden tot de conclusie van dit onderzoek.

\section{Analyse van de cijfers}

Zowel de bekomen sportgegevens als de data uit de bevragingen zullen gebruikt worden om een analyse uit te voeren. Deze analyse zal uiteindelijk leiden tot de conclusie van dit onderzoek.

Tenslotte zal uit deze analyse een discussie volgen over welke aanpassingen aan de POC nog mogelijk zouden zijn en zal belicht worden welke mogelijkheden tot toekomstig onderzoek er zijn.

%% TODO: In dit hoofstuk geef je een korte toelichting over hoe je te werk bent
%% gegaan. Verdeel je onderzoek in grote fasen, en licht in elke fase toe wat
%% de doelstelling was, welke deliverables daar uit gekomen zijn, en welke
%% onderzoeksmethoden je daarbij toegepast hebt. Verantwoord waarom je
%% op deze manier te werk gegaan bent.
%%
%% Voorbeelden van zulke fasen zijn: literatuurstudie, opstellen van een
%% requirements-analyse, opstellen long-list (bij vergelijkende studie),
%% selectie van geschikte tools (bij vergelijkende studie, "short-list"),
%% opzetten testopstelling/PoC, uitvoeren testen en verzamelen
%% van resultaten, analyse van resultaten, ...
%%
%% !!!!! LET OP !!!!!
%%
%% Het is uitdrukkelijk NIET de bedoeling dat je het grootste deel van de corpus
%% van je bachelorproef in dit hoofstuk verwerkt! Dit hoofdstuk is eerder een
%% kort overzicht van je plan van aanpak.
%%
%% Maak voor elke fase (behalve het literatuuronderzoek) een NIEUW HOOFDSTUK aan
%% en geef het een gepaste titel.



