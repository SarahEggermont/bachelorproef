%%=============================================================================
%% Samenvatting
%%=============================================================================

% TODO: De "abstract" of samenvatting is een kernachtige (~ 1 blz. voor een thesis) synthese van het document.
%
% Een goede abstract biedt een kernachtig antwoord op volgende vragen:
%
% 1. Waarover gaat de bachelorproef?
% 2. Waarom heb je er over geschreven?
% 3. Hoe heb je het onderzoek uitgevoerd?
% 4. Wat waren de resultaten? Wat blijkt uit je onderzoek?
% 5. Wat betekenen je resultaten? Wat is de relevantie voor het werkveld?
%
% Daarom bestaat een abstract uit volgende componenten:
%
% - inleiding + kaderen thema
% - probleemstelling
% - (centrale) onderzoeksvraag
% - onderzoeksdoelstelling
% - methodologie
% - resultaten (beperk tot de belangrijkste, relevant voor de onderzoeksvraag)
% - conclusies, aanbevelingen, beperkingen



%%---------- Samenvatting -----------------------------------------------------
% De samenvatting in de hoofdtaal van het document

\chapter*{\IfLanguageName{dutch}{Samenvatting}{Abstract}}

Beweging speelt een grote rol in zowel de fysieke als de mentale gezondheid van mensen. Bijna één derde van de wereldbevolking beweegt te weinig en ondervindt hier vroeg of laat de nadelen van. Om die reden bespreekt dit onderzoek of een sportplatform, dat gebruik maakt van gamification, gebruikt kan worden om medewerkers van \href{https://en.joule.be/}{Joule}, \href{https://www.ventures4growth.com/en}{Ventures 4 Growth}, \href{https://www.mace-legal.com/}{mace}, \href{https://planetb.life/en}{PlanetB}, \href{https://www.we-are.be/}{we are} en \href{https://www.delaware.pro/en-be}{delaware}, die een sedentaire job beoefenen, aan te zetten om meer te sporten.

Na een literatuurstudie rond het belang van beweging, gamification en hoe gamification de intrinsieke motivatie om te bewegen kan bevorderen, worden werknemers van eerder genoemde bedrijven geïnterviewd om de succescriteria en de benodigdheden van het nieuwe platform te bepalen. Aan de hand van deze criteria wordt een platform, in de vorm van een responsive website, gecreëerd. In deze proof of concept wordt gamification geïmplementeerd en worden sportgegevens van deelnemende werknemers verzameld en grafisch voorgesteld op het platform. Tegelijkertijd wordt ook de beleving omtrent het gamification-aspect bevraagd. Deze gegevens leiden, samen met de eerder verzamelde data, tot de conclusie van dit onderzoek.

Dit onderzoek suggereert dat een sportplatform met gamification wel degelijk zorgt voor een verbetering in de hoeveelheid beweging van mensen in een sedentaire job. Daarnaast stelt het dat vooral gamificationtechnieken met een focus op persoonlijke vooruitgang succesvol zijn. Tenslotte zijn er in dit onderzoek geen gamificationtechnieken opgemerkt die een negatief effect hebben op de hoeveelheid beweging, maar moet wel opgemerkt worden dat gebruikers het storend vinden om meldingen te ontvangen als deel van de gamification.