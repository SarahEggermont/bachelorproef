%%=============================================================================
%% Samenvatting
%%=============================================================================

% TODO: De "abstract" of samenvatting is een kernachtige (~ 1 blz. voor een thesis) synthese van het document.
%
% Een goede abstract biedt een kernachtig antwoord op volgende vragen:
%
% 1. Waarover gaat de bachelorproef?
% 2. Waarom heb je er over geschreven?
% 3. Hoe heb je het onderzoek uitgevoerd?
% 4. Wat waren de resultaten? Wat blijkt uit je onderzoek?
% 5. Wat betekenen je resultaten? Wat is de relevantie voor het werkveld?
%
% Daarom bestaat een abstract uit volgende componenten:
%
% - inleiding + kaderen thema
% - probleemstelling
% - (centrale) onderzoeksvraag
% - onderzoeksdoelstelling
% - methodologie
% - resultaten (beperk tot de belangrijkste, relevant voor de onderzoeksvraag)
% - conclusies, aanbevelingen, beperkingen



%%---------- Samenvatting -----------------------------------------------------
% De samenvatting in de hoofdtaal van het document

\chapter*{\IfLanguageName{dutch}{Samenvatting}{Abstract}}

Beweging speelt een grote rol in zowel de fysieke als de mentale gezondheid van mensen. Bijna één derde van de wereldbevolking beweegt te weinig en ondervindt hier vroeg of laat de nadelen van. Om die reden bespreekt dit onderzoek hoe een sportplatform gebruik kan maken van gamification om medewerkers van \href{https://en.joule.be/}{Joule}, \href{https://www.ventures4growth.com/en}{Ventures 4 Growth}, \href{https://www.mace-legal.com/}{mace}, \href{https://planetb.life/en}{PlanetB}, \href{https://www.we-are.be/}{we are} en \href{https://www.delaware.pro/en-be}{delaware} aan te zetten om meer te sporten. Bijkomend kan dit er voor zorgen dat de productiviteit positief beïnvloed wordt.

Na een literatuurstudie rond gamification en het belang van beweging, zullen werknemers van eerder genoemde bedrijven geïnterviewd worden om de succescriteria en de benodigdheden van het nieuwe platform te bepalen. Aan de hand van deze criteria zal een platform, in de vorm van een responsive website, gecreëerd worden. Hierin wordt gamification geïmplementeerd en zullen sportgegevens van deelnemende werknemers verzameld en grafisch voorgesteld worden op het platform. Tegelijkertijd zal ook de beleving omtrent het gamification-aspect bevraagd worden. Deze gegevens zullen na een testperiode geanalyseerd worden om te onderzoeken of er een positieve evolutie merkbaar is in de hoeveelheid beweging van de werknemers en of deze ook effectief aan het platform te danken is.

TODO: hier nog effectieve conclusie noteren.