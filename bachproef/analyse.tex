\chapter{\IfLanguageName{dutch}{Analyse van de cijfers}{Analysis of the figures}}%
\label{ch:analyse}

Allereerst werden de deelnemers van het onderzoek online bevraagd via een vragenlijst. Daarbij werd vooral gepeild naar:
\begin{itemize}
    \item hoe tevreden de deelnemers zijn over hun hoeveelheid beweging,
    \item hoeveel ze bewegen,
    \item wat hen in de weg staat om meer te bewegen,
    \item welk gevoel gamification bij hen teweegbrengt,
    \item wat belangrijk is voor hen in een sportapplicatie.
\end{itemize}

Daarna is de POC, zoals beschreven in hoofdstuk \ref{ch:proofofconcept}, in gebruik genomen door diezelfde mensen. Tenslotte is nogmaals een vragenlijst uitgestuurd om eventuele veranderingen of evoluties op te merken en te analyseren.

De éénentwintig bevraagde mensen zijn allen werkzaam in een sedentaire job. 42.9\% van de respondenten zijn tussen de 18 en 25 jaar oud, 42.9\% tussen de 25 en 35 jaar oud en de overige 14.3\% zijn tussen de 35 en 45 jaar oud (figuur \ref{fig:leeftijd}). Deze bachelorproef kan daarom geen conclusies trekken over mensen die zich buiten deze leeftijdsgroepen bevinden. Gezien de omvang van de steekproef, kan deze bachelorproef suggesties doen die een aanzet geven naar verder onderzoek.

\begin{figure}
    \caption[Verdeling van de leeftijd van deelnemers]{Verdeling van de leeftijd van deelnemers.}
    \includegraphics[width=1\textwidth]{LeeftijdVerdeling}
    \label{fig:leeftijd}
\end{figure}

\section{Bevraging voor het gebruik van ``Move-it!''}

\subsection{Beweging}
Slechts 33.3\% van de bevraagde personen is tevreden met diens hoeveelheid dagelijkse beweging (zie figuur \ref{fig:dagelijkseBeweging}).

\begin{figure}
    \caption[In welke mate vinden gebruikers van zichzelf dat ze dagelijks voldoende bewegen, voor het gebruiken van ``Move-it!'']{In welke mate vinden gebruikers van zichzelf dat ze dagelijks voldoende bewegen, voor het gebruiken van ``Move-it!'' (op een schaal van 1 tot en met 5).}
    \includegraphics[width=1\textwidth]{DailyMovement}
    \label{fig:dagelijkseBeweging}
\end{figure}

Volgens de WHO moeten volwassenen tussen de 18 en 64 jaar oud wekelijks ongeveer 150 à 300 minuten met gemiddelde intensiteit sporten \autocite{Bull2020}.

Uit deze vragenlijst is gebleken dat 47.5\% van de 18 à 45 jarigen dit voorgeschreven aantal niet halen (figuur \ref{fig:gemiddeldSporten}). Dit wil zeggen dat 52.5\% van de mensen dit aangeraden aantal behaalt. 19.1\% van hen sport zelfs meer dan 5 uur per week met gemiddelde intensiteit. Gezien 52.5\% het minimum aantal, voorgeschreven door de WHO, behaalt, maar slechts 33.3\% tevreden is met diens hoeveelheid dagelijkse beweging, kan geconcludeerd worden dat mensen strenger zijn voor zichzelf dan de minima van de WHO zijn.

De WHO stelt dat volwassenen van diezelfde leeftijdsgroep ongeveer 75 à 150 minuten met hoge intensiteit moeten sporten \autocite{Bull2020}.
52.4\% van de respondenten behaalt dit doel en 33.5\% van hen doet zelfs meer dan de aangeraden 150 minuten (figuur \ref{fig:intensiefSporten}).

\begin{figure}[h]
    \caption[Hoeveel gebruikers wekelijks met gemiddelde intensiteit sporten, voor het gebruiken van ``Move-it!'']{Hoeveel gebruikers wekelijks met gemiddelde intensiteit sporten.}
    \includegraphics[width=1\textwidth]{SportenGemiddeld}
    \label{fig:gemiddeldSporten}
\end{figure}

\begin{figure}[h]
    \caption[Hoeveel gebruikers wekelijks met hoge intensiteit sporten, voor het gebruiken van ``Move-it!'']{Hoeveel gebruikers wekelijks met hoge intensiteit sporten, voor het gebruiken van ``Move-it!''.}
    \includegraphics[width=1\textwidth]{SportenIntensief}
    \label{fig:intensiefSporten}
\end{figure}

\subsection{Motivatie}

Slechts vier van de éénentwintig mensen zijn tevreden met de hoeveelheid sport die ze op dit moment beoefenen. Dit wil zeggen dat ruim 81\% niet tevreden is en liever meer zou sporten of bewegen (zie figuur \ref{fig:meerBewegen}).

Het grootste probleem voor mensen om meer te sporten, is tijd vinden. Daarnaast staan ook motivatie en wederkerende blessures in de weg. Een minderheid vindt ook geen plezier in het sporten (zie figuur \ref{fig:waarom}).

\begin{figure}[h]
    \caption[Aanduiding van gebruikers of ze al dan niet meer willen sporten, voor het gebruiken van ``Move-it!'']{Aanduiding van gebruikers of ze al dan niet meer willen sporten, voor het gebruiken van ``Move-it!''.}
    \includegraphics[width=1\textwidth]{MeerSporten}
    \label{fig:meerBewegen}
\end{figure}

\begin{figure}[h]
    \caption[Redenen waarom mensen op dit moment minder sporten, voor het gebruiken van ``Move-it!'']{Redenen waarom mensen op dit moment minder sporten, voor het gebruiken van ``Move-it!''.}
    \includegraphics[width=1\textwidth]{Waarom}
    \label{fig:waarom}
\end{figure}

\subsection{Gamification}

Op vlak van het type gamification in het algemeen, is er een duidelijke voorkeur: 76.2\% van de respondenten verkiest een focus op het zien van persoonlijke vooruitgang. In de applicatie werd deze geïmplementeerd in de vorm van twee grafieken, zoals te zien is op figuur \ref{fig:graphs}.

Dit kan ook teruggevonden worden in wat belangrijk gevonden wordt in een sportapplicatie. 95.2\% vindt het meten van deze persoonlijke vooruitgang, binnen een specifieke sportcontext, namelijk belangrijk en 71.4\% vindt ook technische gegevens zien over de activiteit een meerwaarde.

\subsubsection{Sociale aspecten}

Het valt ook op dat sociale interacties, zoals bijvoorbeeld likes, comments en volgers, net als storend ervaren worden door 47.6\%. Dit is tegenstrijdig met wat de literatuur stelt. Daar wordt beschreven hoe gebruikers enerzijds sociale feedback, en anderzijds sociale invloed nodig hebben (zie subsectie \ref{ssec:sociale_aspecen}). Een mogelijke verklaring hiervoor zou kunnen zijn dat dit wordt ervaren als sociale druk, wat de deelnemers angstgevoelens kan bezorgen, uit vrees om niet aan de verwachtingen te voldoen \autocite{Jong2010}. Dit onderzoek heeft ervoor gekozen om geen gebruik te maken van sociale interacties en gaat daarom niet op zoek naar een bevestiging of ontkrachting van deze potentiële achterliggende reden.

\subsubsection{Psychologische aspecten: motivatie}

Het is daarnaast ook tegenstrijdig dat 38.1\% uitgedaagd en gemotiveerd wil worden door de applicatie, terwijl 42.9\% het storend vindt meldingen te ontvangen van een sportapplicatie. Dit wil zeggen dat gebruikers van deze applicatie op andere manieren dan gewoonlijk gemotiveerd willen worden.
De huidige meest voorkomende vorm van motivatie gebeurt namelijk door middel van notificaties:
Strava motiveert bijvoorbeeld door meldingen te sturen wanneer iemand van je sociale kring een nieuwe activiteit post, je voorbijgestoken wordt in een klassement of wanneer je een doel bereikt.
Welke alternatieven bestaan voor deze notificaties, die niet als storend ervaren worden en wel motiverend werken, is een vraag die uitnodigt naar verder onderzoek.

\subsection{Meest gebruikte sportapplicaties}

Strava is de populairste sportapplicatie binnen deze steekproef: 85.7\% van de deelnemers geeft aan Strava te gebruiken. Daarna volgen Garmin Sports (28.6\%), Adidas Runtastic (19\%), Apple Workouts (19\%) en Nike Run Club (9.5\%). Niemand uit deze steekproef gebruikt Fitbit. Er zijn ook 2 personen die geen sportapplicaties gebruiken.

\section{Sportgegevens ``Move-it!''}

Op figuren \ref{fig:evolutie1}, \ref{fig:evolutie2}, \ref{fig:evolutie3}, \ref{fig:evolutie4}, \ref{fig:evolutie5}, \ref{fig:evolutie6}, \ref{fig:evolutie7} en \ref{fig:evolutie8} wordt de evolutie weergegeven van een deel van de deelnemers van dit onderzoek. Deze zijn uit de steekproef gekozen door de consistentie waarmee ze data hebben ingegeven. Van andere deelnemers is het minder waarschijnlijk dat de ingevoerde data een realistische representatie is. Het valt hierbij op dat vooral jongere mensen (van 22 tot en met 28 jaar) gemotiveerd waren om sportgegevens in te vullen. In de volgende sectie wordt besproken wat hier mogelijke oorzaken van kunnen zijn.

Bij vijf van de acht deelnemers, is er een stijging waarneembaar in de hoeveelheid dagelijkse beweging. Drie personen bleven met dezelfde intensiteit en frequentie bewegen. In het volgende deel van dit onderzoek wordt bevraagd of deze stijging al dan niet te maken heeft met de ingebruikname van het sportplatform.

\begin{figure}[htbp]
    \begin{minipage}[t]{0.48\linewidth} % adjust width as needed
        \centering
        \caption[Evolutie van de activiteit van een 27-jarige]{Evolutie van de activiteit ingegeven in ``Move-it!'' van een 27-jarige.}
        \includegraphics[width=1\textwidth]{GrafiekEvolutie1}
        \label{fig:evolutie1}
    \end{minipage}
    \hfill
    \begin{minipage}[t]{0.48\linewidth} % adjust width as needed
        \centering
        \caption[Evolutie van de activiteit van een 22-jarige]{Evolutie van de activiteit ingegeven in ``Move-it!'' van een 22-jarige.}
        \includegraphics[width=1\textwidth]{GrafiekEvolutie2}
        \label{fig:evolutie2}
    \end{minipage}
\end{figure}

\begin{figure}[htbp]
    \begin{minipage}[t]{0.48\linewidth} % adjust width as needed
        \centering
        \caption[Evolutie van de activiteit van een 25-jarige]{Evolutie van de activiteit ingegeven in ``Move-it!'' van een 25-jarige.}
        \includegraphics[width=1\textwidth]{GrafiekEvolutie3}
        \label{fig:evolutie3}
    \end{minipage}
    \hfill
    \begin{minipage}[t]{0.48\linewidth} % adjust width as needed
        \centering
        \caption[Evolutie van de activiteit van een 28-jarige]{Evolutie van de activiteit ingegeven in ``Move-it!'' van een 28-jarige.}
        \includegraphics[width=1\textwidth]{GrafiekEvolutie4}
        \label{fig:evolutie4}
    \end{minipage}
\end{figure}
\begin{figure}[htbp]
    \begin{minipage}[t]{0.48\linewidth} % adjust width as needed
        \centering
        \caption[Evolutie van de activiteit van een 25-jarige]{Evolutie van de activiteit ingegeven in ``Move-it!'' van een 25-jarige.}
        \includegraphics[width=1\textwidth]{GrafiekEvolutie5}
        \label{fig:evolutie5}
    \end{minipage}
    \hfill
    \begin{minipage}[t]{0.48\linewidth} % adjust width as needed
        \centering
        \caption[Evolutie van de activiteit van een 27-jarige]{Evolutie van de activiteit ingegeven in ``Move-it!'' van een 27-jarige.}
        \includegraphics[width=1\textwidth]{GrafiekEvolutie6}
        \label{fig:evolutie6}
    \end{minipage}
\end{figure}
\begin{figure}[htbp]
    \begin{minipage}[t]{0.48\linewidth} % adjust width as needed
        \centering
        \caption[Evolutie van de activiteit van een 28-jarige]{Evolutie van de activiteit ingegeven in ``Move-it!'' van een 28-jarige.}
        \includegraphics[width=.9\textwidth]{GrafiekEvolutie7}
        \label{fig:evolutie7}
    \end{minipage}
    \hfill
    \begin{minipage}[t]{0.48\linewidth} % adjust width as needed
        \centering
        \caption[Evolutie van de activiteit van een 22-jarige]{Evolutie van de activiteit ingegeven in ``Move-it!'' van een 22-jarige.}
        \includegraphics[width=.9\textwidth]{GrafiekEvolutie8}
        \label{fig:evolutie8}
    \end{minipage}
\end{figure}

Daarnaast kan uit de sportgegevens ook het gemiddeld aantal minuten beweging per dag en per week gehaald worden, met als doel te analyseren of de stijging in de hoeveelheid dagdagelijkse beweging zich heeft doorgezet tot het aangeraden niveau van de WHO. Hiervoor worden opnieuw de gegevens gebruikt van de acht proefpersonen waarvoor grafieken gegenereerd werden.

\begin{table}[h]
    \caption[Gemiddeld aantal minuten beweging per dag en per week]{Gemiddeld aantal minuten beweging per dag en per week van de meest consistente deelnemers.}
    \centering
    \label{table:gemiddeldes}
\begin{tabular}{||m{.2\linewidth} m{.4\linewidth} m{.4\linewidth}||}
    \hline
    Leeftijd & Gemiddeld aantal minuten beweging per dag & Gemiddeld aantal minuten beweging per week \\ [0.5ex]
    \hline\hline
    22 & 59.75 & 418.25 \\
    22 & 29.35 & 205.45 \\
    25 & 47 & 329 \\
    25 & 32.13 & 224.91 \\
    27 & 63.96 & 447.72 \\
    27 & 35 & 245 \\
    28 & 32.86 & 230.02 \\
    28 & 74.22 & 519.54 \\ [1ex]
    \hline
\end{tabular}
\end{table}

De cijfers uit tabel \ref{table:gemiddeldes} geven het gemiddeld aantal minuten beweging per dag en per week terug, berekend op basis van de data die gedurende vier weken verzameld is.
Deze suggereren dat jongeren tussen de 22 en 28 jaar de vooropgestelde bewegingsnormen van de WHO wel degelijk behalen.
Het is daarbij wel zeer belangrijk te vermelden dat dit onderzoek geen onderscheid kan maken tussen sportieve activiteiten met gemiddelde of intensieve inzet. Daarom ziet het de bekomen data als de som van de gemiddeld intense en erg intensieve activiteiten. Als minimum wordt daarom ook de som genomen, wat neerkomt op 225 à 450 minuten per week. Dit specifiek aspect moet verder besproken worden in een onderzoek waarin elke proefpersoon een hartslagmonitor ter beschikking krijgt, waardoor een onderscheid zal kunnen gemaakt worden tussen beide types activiteiten.

\section{Bevraging na het gebruik van ``Move-it!''}

\subsection{Beweging}

Voor aanvang van het gebruik van ``Move-it!'', gaf 81\% van de deelnemers aan meer te willen sporten. Na het in gebruik nemen van het sportplatform, is dit gedaald naar 58.3\%.
De helft van de deelnemers geeft ook aan tevreden te zijn met diens hoeveelheid dagelijkse beweging, ten opzichte van de 19\% voordien.

Daarbovenop geven 58.4\% van de deelnemers aan dat ze meer dan 150 minuten per week bewegen met gemiddelde intensiteit, wat een stijging is ten opzichte van de 47.5\% voordien. Voor sporten met hoge intensiteit, is een stijging merkbaar in het aantal mensen die 75 a 150 minuten beoefenen per week: 52.4\% is gestegen naar 66.7\%. Een daling is echter merkbaar in het aantal mensen dat meer dan deze 150 minuten per week intensief sportte (33.5\%), dit is nu 24.9\%.

\subsubsection{Vergelijking met de sportgegevens}

Wanneer de sportgegevens vergeleken worden met de antwoorden uit de laatste bevraging, kan opgemerkt worden dat deze in dezelfde lijn liggen. De helft geeft in de vragenlijst namelijk aan het gevoel te hebben meer te bewegen sinds het gebruik van ``Move-it!'', terwijl de andere helft aangeeft ongeveer even veel te bewegen (zie figuur \ref{fig:evolutie_beweging}). Voor de sportgegevens wordt ook een stijging waargenomen bij 62.5\% van de deelnemers, wat wil zeggen dat 12.5\% zichzelf lager inschat.

\begin{figure}[h]
    \caption[Evolutie in de hoeveelheid beweging, sinds het gebruiken van ``Move-it!'']{Evolutie in de hoeveelheid beweging, sinds het gebruiken van ``Move-it!''.}
    \includegraphics[width=1\textwidth]{VeranderingBeweging}
    \label{fig:evolutie_beweging}
\end{figure}

Deze cijfers suggereren dat de bevindingen van \textcite{Kari2016} en \textcite{Bitrian2020}, die eerder al aantoonden dat gamification in sportapplicaties een positieve invloed heeft op de intrinsieke bewegingsmotivatie, ook toegepast mogen worden op mensen met een sedentair beroep.

\subsection{Motivatie}

Bij het in gebruik nemen van het sportplatform, viel het op dat niet iedereen even consistent was met het ingeven van data. Een mogelijke oorzaak hiervan zou te vinden zijn bij de competitiviteit van de deelnemers. Deze applicatie gebruikt namelijk scoreborden en punten als gamification techniek, welke een bepaalde competitie creëren (zie \ref{sssec:labels_scoreborden}). Niet- of weinig-competitieve gebruikers zullen hier echter geen of in mindere mate invloed van ondervinden.

Uit de bevraging op het einde van de testperiode van ``Move-it!'', is gebleken dat van alle deelnemers 8.3\% niet en 8.3\% slechts een beetje competitief ingesteld is (zie figuur \ref{fig:competitief}). Dit verklaart vermoedelijk een deel van de inconsistentie waarmee sommigen gegevens ingaven.

\begin{figure}[h]
    \caption[Competitieve aard van gebruikers van ``Move-it!'']{Competitieve aard van gebruikers van ``Move-it!''.}
    \includegraphics[width=1\textwidth]{Competitief}
    \label{fig:competitief}
\end{figure}

Wat ook in beschouwing moet worden genomen, is dat sommige deelnemers, die uit zichzelf al regelmatig intensief sportten, mogelijks niet al hun sportgegevens ingaven. Door het uitsturen van herinneringsmails, viel het op dat personen die reeds activiteiten ingegeven hadden, vooral gemiddeld intensieve activiteiten aanvulden. Het kan bijvoorbeeld zo zijn dat zeer fitte mensen woon-werkverkeer niet beschouwen als ``echt'' bewegen, omdat hun lichaam een veel hogere intensiteit gewoon is. Dit is nog een vraag die uitnodigt naar verder onderzoek.

Een andere mogelijke oorzaak voor de inconsistentie, zou ook het manueel ingeven van de gegevens kunnen zijn, gezien dit ook enkele keren terugkwam in de suggesties om het platform beter te maken (zie ook \ref{ssec:verbeterpunten}).

Het valt daarnaast op dat vooral de oudste deelnemers in deze steekproef niet consistent waren met het ingeven van data, wat de bewering van \textcite{PoloPena2020}, dat ouderen gevoeliger zijn voor gamification, lijkt tegen te spreken.

Voor mensen die meer wensen te sporten, werd er opnieuw bevraagd wat hen tegenhoudt. Alle deelnemers gaven aan dat tijd vinden een probleem is. In vergelijking met voor het gebruik van ``Move-it!'', is het wel opvallend dat nog maar 1 persoon geen motivatie vindt om te bewegen, ten opzichte van de 6 voordien.

\subsection{Gamification}

Het populairste deel van de gamification, is de persoonlijke ranking (zie \ref{fig:personalRanking}) met alle sporters: 83.3\% van de deelnemers geeft aan dit als motiverend te ervaren.
Daarna volgt het opvolgen van de eigen vooruitgang door middel van de grafieken op het dashboard (zie \ref{fig:graphs}) met 58.3\%.
Op de laatste plaats staat de lijst van de eigen gepresteerde activiteiten (zie \ref{fig:performances}) met 41.7\%.

Opvallend is dat gamification die focust op persoonlijke vooruitgang op voorhand als winnaar uit de bus kwam, terwijl het nu slechts op de tweede en derde plek belandt.

\subsubsection{Psychologische aspecten: motivatie}

Om gebruikers de weg te helpen vinden naar het sportplatform, werden wekelijks mails uitgestuurd naar minder actieve gebruikers, ter herinnering om hun data in te geven en ter motivatie. Dit had echter niet altijd het gewenste effect.
Mogelijks is dit te linken aan de 42.9\% die het storend vindt om meldingen te ontvangen van sportapplicaties.

\subsection{Verbeterpunten van het sportplatform}
\label{ssec:verbeterpunten}

Wanneer bevraagd wordt hoe een sportapplicatie kan helpen om meer te bewegen, kwamen volgende zaken aan bod:

\begin{itemize}
    \item een trainingsschema voorzien,
    \item inspiratie bieden op vlak van oefeningen of soorten bewegen, om zo variatie te brengen,
    \item een sociaal gebeuren creëren, om zo gelijk gezinde mensen te vinden,
    \item een realistische en flexibele planning kunnen opstellen, aangepast aan de eigen (soms zeer drukke) agenda, waarin trainingen verplaatst kunnen worden.
\end{itemize}

Tijdens de laatste bevraging werd er ook ruimte gelaten voor suggesties om het sportplatform nog te verbeteren, hierbij geeft 54.5\% aan dat een bepaalde verandering er voor zou kunnen zorgen dat desbetreffende deelnemer het sportplatform vaker zou gebruiken. De meest voorkomende verbeterpunten zijn:

\begin{itemize}
  \item Het mogelijk maken om sportprestaties te bewerken.
  \item Het makkelijker maken om sportgegevens in te geven, door bijvoorbeeld integratie met Strava.
  \item Scores bepalen aan de hand van de intensiteit van de activiteiten.
\end{itemize}