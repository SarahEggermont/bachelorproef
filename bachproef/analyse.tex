\chapter{\IfLanguageName{dutch}{Analyse van de cijfers}{Analysis of the figures}}%
\label{ch:analyse}

Allereerst werden de deelnemers van het onderzoek online bevraagd via een vragenlijst. Daarna is de POC in gebruik genomen door diezelfde mensen als zij die de vragenlijst ingevuld hebben. Tenslotte is nogmaals een vragenlijst uitgestuurd om eventuele veranderingen of evoluties op te merken.

De éénentwintig bevraagde mensen zijn allen werkzaam in een sedentaire job. 42.9\% van de respondenten zijn tussen de 18 en 25 jaar oud, 42.9\% tussen de 25 en 35 jaar oud en de overige 14.3\% zijn tussen de 35 en 45 jaar oud (figuur \ref{fig:leeftijd}). Deze bachelorproef kan dus geen conclusies trekken over mensen die zich buiten deze leeftijdsgroepen bevinden. Gezien de omvang van de steekproef kunnen ook over andere leeftijdsgroepen geen conclusies getrokken worden, wel kunnen suggesties gedaan worden en een aanzet gegeven worden naar verder onderzoek.

\begin{figure}[h]
    \caption[Verdeling van de leeftijd van deelnemers.]{Verdeling van de leeftijd van deelnemers.}
    \includegraphics[width=1\textwidth]{LeeftijdVerdeling}
    \label{fig:leeftijd}
\end{figure}

\section{Resultaten voor gebruik van Move-it!}

\subsection{Beweging}
Slechts 33.3\% van de bevraagde personen is tevreden met diens hoeveelheid dagelijkse beweging (zie figuur \ref{fig:dagelijkseBeweging}).

\begin{figure}[h]
    \caption[In welke mate vindt u van uzelf dat u dagelijks voldoende beweegt?]{In welke mate vindt u van uzelf dat u dagelijks voldoende beweegt (op een schaal van 1 tot en met 5)?}
    \includegraphics[width=1\textwidth]{DailyMovement}
    \label{fig:dagelijkseBeweging}
\end{figure}

Volgens de WHO moeten volwassenen tussen de 18 en 64 jaar oud wekelijks ongeveer 150 à 300 minuten met gemiddelde intensiteit sporten \autocite{Bull2020}.
Uit deze vragenlijst is gebleken dat 47.5\% van de 18 à 45 jaar oude mensen dit voorgeschreven aantal niet halen. 19.1\% van hen sport meer dan 5 uur per week met gemiddelde intensiteit. Dit wil zeggen dat slechts 33.4\% van de mensen dit aangeraden aantal behaalt.

\begin{figure}[h]
    \caption[Hoeveel sport u dagelijks met gemiddelde intensiteit?]{Hoeveel sport u dagelijks met gemiddelde intensiteit?}
    \includegraphics[width=1\textwidth]{SportenGemiddeld}
    \label{fig:gemiddeldSporten}
\end{figure}


WHO stelt dat volwassenen van diezelfde leeftijdsgroep ongeveer 75 à 150 minuten met hoge intensiteit moeten sporten \autocite{Bull2020}.
52.4\% van de respondenten behaalt dit doel en 33.5\% van hen doet zelfs meer dan de aangeraden 150 minuten.

\begin{figure}[h]
    \caption[Hoeveel sport u dagelijks met hoge intensiteit?]{Hoeveel sport u dagelijks met hoge intensiteit?}
    \includegraphics[width=1\textwidth]{SportenIntensief}
    \label{fig:intensiefSporten}
\end{figure}


\subsection{Motivatie}

\begin{figure}[h]
    \caption[Zou u liever meer sporten dan u op dit moment doet?]{Zou u liever meer sporten dan u op dit moment doet?}
    \includegraphics[width=1\textwidth]{MeerSporten}
    \label{fig:meerBewegen}
\end{figure}

Slechts vier van de éénentwintig mensen zijn tevreden met de hoeveelheid sport die ze op dit moment doen. Ruim 81\% is dus niet tevreden en zou liever meer sporten of bewegen (zie figuur \ref{fig:meerBewegen}).

Het grootste probleem voor mensen om meer te sporten, is tijd vinden. Daarnaast staan ook motivatie en wederkerende blessures in de weg. Een minderheid vindt ook geen plezier in het sporten (zie figuur \ref{fig:waarom}).

\begin{figure}[h]
    \caption[Wat houdt u op dit moment tegen om meer te sporten?]{Wat houdt u op dit moment tegen om meer te sporten?}
    \includegraphics[width=1\textwidth]{Waarom}
    \label{fig:waarom}
\end{figure}

\subsection{Gamification}

Op vlak van het type gamification, is er een duidelijke voorkeur: 76.2\% van de respondenten verkiest een focus op persoonlijke vooruitgang.

Dit kan ook teruggevonden worden in wat belangrijk gevonden wordt in een sportapplicatie. 95.2\% vindt het meten van persoonlijke vooruitgang namelijk belangrijk en 71.4\% vindt ook technische gegevens zien over de activiteit een meerwaarde.

Het valt ook op dat sociale interacties, zoals likes, comments en volgers bijvoorbeeld, net als storend ervaren worden door 47.6\%, dit is tegenstrijdig met de literatuur. Dit zou te wijten kunnen zijn aan de doelgroep van dit onderzoek: mogelijks willen deelnemers geen sociale druk voelen, die als negatief ervaren kan worden.

Nog een tegenstrijdigheid die opvalt, is dat 38.1\% uitgedaagd en gemotiveerd wil worden door de applicatie, maar dat 42.9\% het storend vindt meldingen te ontvangen van een sportapplicatie. Dit wil dus zeggen dat deze motivatie op andere manieren dan gewoonlijk opgewekt moet worden.

\section{Sportresultaten}

Op figuren \ref{fig:evolutie1}, \ref{fig:evolutie2}, \ref{fig:evolutie3}, \ref{fig:evolutie4}, \ref{fig:evolutie5}, \ref{fig:evolutie6}, \ref{fig:evolutie7} en \ref{fig:evolutie8} wordt de evolutie weergegeven van een deel van de deelnemers van dit onderzoek. Deze zijn uit de steekproef gekozen door de consistentie waarmee ze data hebben ingegeven. Van andere deelnemers is het minder waarschijnlijk dat de ingevoerde data een realistische representatie is. Het valt hierbij op dat vooral jongere mensen (van 22 tot en met 28 jaar) gemotiveerd waren om sportgegevens in te vullen. In de volgende sectie wordt besproken wat hier mogelijke oorzaken van kunnen zijn.

Bij vijf van de acht deelnemers, is er een stijging waarneembaar in de hoeveelheid dagelijkse beweging. Drie personen bleven met dezelfde intensiteit en frequentie bewegen. In het volgende deel van dit onderzoek wordt bevraagd of deze stijging al dan niet te maken heeft met de ingebruikname van het sportplatform.

\begin{figure}[htbp]
    \begin{minipage}[t]{0.45\linewidth} % adjust width as needed
        \centering
        \caption[Evolutie van de activiteit van een 27-jarige.]{Evolutie van de activiteit ingegeven in Move-it van een 27-jarige.}
        \includegraphics[width=1\textwidth]{GrafiekEvolutie1}
        \label{fig:evolutie1}
    \end{minipage}
    \hfill
    \begin{minipage}[t]{0.45\linewidth} % adjust width as needed
        \centering
        \caption[Evolutie van de activiteit van een 22-jarige.]{Evolutie van de activiteit ingegeven in Move-it van een 22-jarige.}
        \includegraphics[width=1\textwidth]{GrafiekEvolutie2}
        \label{fig:evolutie2}
    \end{minipage}
\end{figure}

\begin{figure}[htbp]
    \begin{minipage}[t]{0.45\linewidth} % adjust width as needed
        \centering
        \caption[Evolutie van de activiteit van een 25-jarige.]{Evolutie van de activiteit ingegeven in Move-it van een 25-jarige.}
        \includegraphics[width=1\textwidth]{GrafiekEvolutie3}
        \label{fig:evolutie3}
    \end{minipage}
    \hfill
    \begin{minipage}[t]{0.45\linewidth} % adjust width as needed
        \centering
        \caption[Evolutie van de activiteit van een 28-jarige.]{Evolutie van de activiteit ingegeven in Move-it van een 28-jarige.}
        \includegraphics[width=1\textwidth]{GrafiekEvolutie4}
        \label{fig:evolutie4}
    \end{minipage}
\end{figure}
\begin{figure}[htbp]
    \begin{minipage}[t]{0.45\linewidth} % adjust width as needed
        \centering
        \caption[Evolutie van de activiteit van een 25-jarige.]{Evolutie van de activiteit ingegeven in Move-it van een 25-jarige.}
        \includegraphics[width=1\textwidth]{GrafiekEvolutie5}
        \label{fig:evolutie5}
    \end{minipage}
    \hfill
    \begin{minipage}[t]{0.45\linewidth} % adjust width as needed
        \centering
        \caption[Evolutie van de activiteit van een 27-jarige.]{Evolutie van de activiteit ingegeven in Move-it van een 27-jarige.}
        \includegraphics[width=1\textwidth]{GrafiekEvolutie6}
        \label{fig:evolutie6}
    \end{minipage}
\end{figure}
\begin{figure}[htbp]
    \begin{minipage}[t]{0.45\linewidth} % adjust width as needed
        \centering
        \caption[Evolutie van de activiteit van een 28-jarige.]{Evolutie van de activiteit ingegeven in Move-it van een 28-jarige.}
        \includegraphics[width=.9\textwidth]{GrafiekEvolutie7}
        \label{fig:evolutie7}
    \end{minipage}
    \hfill
    \begin{minipage}[t]{0.45\linewidth} % adjust width as needed
        \centering
        \caption[Evolutie van de activiteit van een 22-jarige.]{Evolutie van de activiteit ingegeven in Move-it van een 22-jarige.}
        \includegraphics[width=.9\textwidth]{GrafiekEvolutie8}
        \label{fig:evolutie8}
    \end{minipage}
\end{figure}

Daarnaast kunnen uit de sportgegevens ook het gemiddeld aantal minuten beweging per dag en per week gehaald worden. Hiervoor worden opnieuw de gegevens gebruikt van de acht proefpersonen waarvoor grafieken gegenereerd werden.

\begin{figure}[h]
    \caption[Gemiddeld aantal minuten beweging per dag en per week.]{Gemiddeld aantal minuten beweging per dag en per week van de meest consistente deelnemers.}
    \centering
    \label{table:gemiddeldes}
\begin{tabular}{||m{.2\textwidth} m{.4\textwidth} m{.4\textwidth}||}
    \hline
    Leeftijd & Gemiddeld aantal minuten beweging per dag & Gemiddeld aantal minuten beweging per week \\ [0.5ex]
    \hline\hline
    22 & 59.75 & 418.25 \\
    22 & 29.35 & 205.45 \\
    25 & 47 & 329 \\
    25 & 32.13 & 224.91 \\
    27 & 63.96 & 447.72 \\
    27 & 35 & 245 \\
    28 & 32.86 & 230.02 \\
    28 & 74.22 & 519.54 \\ [1ex]
    \hline
\end{tabular}
\end{figure}

De cijfers uit tabel \ref{table:gemiddeldes} suggereren dat jongeren tussen de 22 en 28 jaar de vooropgestelde bewegingsnormen van de WHO wel degelijk behalen.
Het is daarbij wel zeer belangrijk te vermelden dat dit onderzoek geen onderscheid kan maken tussen sportieve activiteiten met gemiddelde of intensieve inzet. Dit moet  verder onderzocht worden in een onderzoek waarin elke proefpersoon een hartslagmonitor ter beschikking krijgt, waardoor een onderscheid zal kunnen gemaakt worden tussen beide types activiteiten.

Tenslotte moet volgende bemerking gemaakt worden: geven deelnemers die uit zichzelf regelmatig intensief sporten niet al hun bewegingsgegevens in of bewegen ze effectief weinig daarnaast? Dit is een vraag die uitnodigen naar verder onderzoek.

\section{Resultaten na gebruik van Move-it!}

OPMERKING: voor het uitschrijven van deze sectie, wacht ik nog op de laatste resultaten van mijn vragenlijst. Hieronder kunt u wel reeds een skelet vinden van wat hier besproken zal worden.

Allereerst wordt bevraagd hoe competitief ingesteld mensen zijn. Dit zal gekoppeld worden aan het al dan niet consistent invoeren van gegevens op het sportplatform.

Daarna moeten gebruikers inschatten of ze van zichzelf vinden dat ze meer of minder gesport hebben sinds het gebruik van het sportplatform, dit kan vergeleken worden met de effectieve resultaten.

Tenslotte wordt bevraagd welke gamification in het sportplatform de gebruikers het meeste aansprak en hoe het sportplatform eventueel verbeterd zou kunnen worden.
